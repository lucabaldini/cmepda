\documentclass[9pt]{beamer}
\usetheme{cmepda}

\usepackage[utf8]{inputenc}
\usepackage[T1]{fontenc}
% Remove the prefix 'Figure' from captions
\usepackage{caption}
\captionsetup[figure]{labelformat=empty}% redefines the caption setup of the figures environment in the beamer class.

\graphicspath{{figures/}} 


\title{Advanced python features}
\subtitle{Computing Methods for Experimental Physics and Data Analysis}
\date{Compiled on \today}
\author{A. Manfreda}
\institute[INFN]{INFN--Pisa}
\email{alberto.manfreda@pi.infn.it}


\begin{document}

\titleframe

\begin{frame}
  \frametitle{Errors and Exceptions}
  \begin{itemize}
    \item \alert{Error handling} is one of the most important problem to solve when designing a program
    \item What should I do when I piece of code fails?
    \item What does fail mean?
    \begin {itemize}
      \item Invalid input e.g. passing a path to a non existent file, or passing a string to a function for dividing numbers
      \item Valid output not found, e.g searching the position of the letter 'd' in the string 'elephant'
      \item Output cannot be find in a reasonable amount of time
      \item Runtime resource failures: network connection down, disk space ended\dots
    \end{itemize}
    \item Two phylosophies (historically):
    \begin{itemize}
      \item Return some \alert{error flag} (in different ways) to tell the user that something went wrong
      \item \alert{Exceptions}
    \end{itemize}
    \item Example: a typical convention for programs is to return 0 from the main if the execution was successful and an
          error code (integer number) otherwise
  \end{itemize}
\end{frame}


\begin{frame}
  \frametitle{Error flags}
  \begin{Verbatim}[label=\makebox{\url{https://github.com/lucabaldini/cmepda/tree/master/slides/latex/snippets/error\_flags.py}},commandchars=\\\{\}]
\PY{c+c1}{\PYZsh{} The \PYZsq{}find()\PYZsq{} method for strings in python uses an error flag}
\PY{n}{text} \PY{o}{=} \PY{l+s+s1}{\PYZsq{}}\PY{l+s+s1}{elephant}\PY{l+s+s1}{\PYZsq{}}
\PY{k}{print}\PY{p}{(}\PY{n}{text}\PY{o}{.}\PY{n}{find}\PY{p}{(}\PY{l+s+s1}{\PYZsq{}}\PY{l+s+s1}{p}\PY{l+s+s1}{\PYZsq{}}\PY{p}{)}\PY{p}{)} \PY{c+c1}{\PYZsh{} upon success returns the position of the substring}
\PY{k}{print}\PY{p}{(}\PY{n}{text}\PY{o}{.}\PY{n}{find}\PY{p}{(}\PY{l+s+s1}{\PYZsq{}}\PY{l+s+s1}{d}\PY{l+s+s1}{\PYZsq{}}\PY{p}{)}\PY{p}{)} \PY{c+c1}{\PYZsh{} returns \PYZhy{}1 if the substring is not found}

\PY{c+c1}{\PYZsh{} Why is this dangerous?}
\PY{k}{def} \PY{n+nf}{cut\PYZus{}two\PYZus{}before}\PY{p}{(}\PY{n}{input\PYZus{}string}\PY{p}{,} \PY{n}{substring}\PY{p}{)}\PY{p}{:}
    \PY{l+s+sd}{\PYZdq{}\PYZdq{}\PYZdq{} Cut a string up to two positions before that of the given substring and}
\PY{l+s+sd}{    return it \PYZdq{}\PYZdq{}\PYZdq{}}
    \PY{n}{pos} \PY{o}{=} \PY{n}{input\PYZus{}string}\PY{o}{.}\PY{n}{find}\PY{p}{(}\PY{n}{substring}\PY{p}{)}
    \PY{k}{return} \PY{n}{input\PYZus{}string}\PY{p}{[}\PY{p}{:}\PY{p}{(}\PY{n}{pos}\PY{o}{\PYZhy{}}\PY{l+m+mi}{1}\PY{p}{)}\PY{p}{]}

\PY{c+c1}{\PYZsh{} If the substring exists in the string everything works fine}
\PY{k}{print}\PY{p}{(}\PY{n}{cut\PYZus{}two\PYZus{}before}\PY{p}{(}\PY{l+s+s1}{\PYZsq{}}\PY{l+s+s1}{We all live in a Yellow Submarine}\PY{l+s+s1}{\PYZsq{}}\PY{p}{,} \PY{l+s+s1}{\PYZsq{}}\PY{l+s+s1}{Yellow}\PY{l+s+s1}{\PYZsq{}}\PY{p}{)}\PY{p}{)}
\PY{c+c1}{\PYZsh{} What will be the output here?}
\PY{k}{print}\PY{p}{(}\PY{n}{cut\PYZus{}two\PYZus{}before}\PY{p}{(}\PY{l+s+s1}{\PYZsq{}}\PY{l+s+s1}{We all live in a Yellow Submarine}\PY{l+s+s1}{\PYZsq{}}\PY{p}{,} \PY{l+s+s1}{\PYZsq{}}\PY{l+s+s1}{Red}\PY{l+s+s1}{\PYZsq{}}\PY{p}{)}\PY{p}{)}

[Output]
3
-1
We all live in a
We all live in a Yellow Submari
\end{Verbatim}
\end{frame}


\begin{frame}
  \frametitle{Problems of error flags}
    Error codes have their use (and are fine in some cases) but they suffer from a few issues:
    \begin{itemize}
      \item Choosing them is often arbitrary (and sometimes is difficult to make a sensible choice)
      \begin{itemize}
        \item What if all the numbers can represent meaningful output of the function?
      \end{itemize}
      \item Are cumbersome to use
      \begin{itemize}
        \item Which error flag is used by a function? 0? -1? 99999999? $\rightarrow$ you have to go through the documentation for each!
        \item If you have a deep hierarchy of functions you have to perform checks and pass the error up at every level!
      \end{itemize}
      \item What if the caller of a function does not check the error flag?
      \begin{itemize}
        \item The bug can propagate \alert{silently} through its code!
      \end{itemize}
    \end{itemize}
    \medskip
    We want something that:
    \begin{itemize}
      \item Is clearly separated from the returned output
      \item Cannot be silently ignored by the user
      \item Is easy to report to upper level without lots of lines of code
    \end{itemize}  
\end{frame}


\begin{frame}
  \frametitle{A different way}
  \begin{Verbatim}[label=\makebox{\url{https://github.com/lucabaldini/cmepda/tree/master/slides/latex/snippets/exceptions\_vs\_err\_flags.py}},commandchars=\\\{\}]
\PY{c+c1}{\PYZsh{} index() is the same as find(), but rise an exception in case of failure}
\PY{k}{def} \PY{n+nf}{cut\PYZus{}two\PYZus{}before}\PY{p}{(}\PY{n}{input\PYZus{}string}\PY{p}{,} \PY{n}{substring}\PY{p}{)}\PY{p}{:}
    \PY{l+s+sd}{\PYZdq{}\PYZdq{}\PYZdq{} Cut a string up to two positions before that of the given substring and}
\PY{l+s+sd}{    return it \PYZdq{}\PYZdq{}\PYZdq{}}
    \PY{n}{pos} \PY{o}{=} \PY{n}{input\PYZus{}string}\PY{o}{.}\PY{n}{index}\PY{p}{(}\PY{n}{substring}\PY{p}{)}
    \PY{k}{return} \PY{n}{input\PYZus{}string}\PY{p}{[}\PY{p}{:}\PY{p}{(}\PY{n}{pos}\PY{o}{\PYZhy{}}\PY{l+m+mi}{1}\PY{p}{)}\PY{p}{]}

\PY{c+c1}{\PYZsh{} If the substring exists in the string everything works fine}
\PY{k}{print}\PY{p}{(}\PY{n}{cut\PYZus{}two\PYZus{}before}\PY{p}{(}\PY{l+s+s1}{\PYZsq{}}\PY{l+s+s1}{We all live in a Yellow Submarine}\PY{l+s+s1}{\PYZsq{}}\PY{p}{,} \PY{l+s+s1}{\PYZsq{}}\PY{l+s+s1}{Yellow}\PY{l+s+s1}{\PYZsq{}}\PY{p}{)}\PY{p}{)}
\PY{c+c1}{\PYZsh{} No silent bug here!}
\PY{k}{print}\PY{p}{(}\PY{n}{cut\PYZus{}two\PYZus{}before}\PY{p}{(}\PY{l+s+s1}{\PYZsq{}}\PY{l+s+s1}{We all live in a Yellow Submarine}\PY{l+s+s1}{\PYZsq{}}\PY{p}{,} \PY{l+s+s1}{\PYZsq{}}\PY{l+s+s1}{Red}\PY{l+s+s1}{\PYZsq{}}\PY{p}{)}\PY{p}{)}

[Output]
We all live in a
Traceback (most recent call last):
  File "snippets/exceptions_vs_err_flags.py", line 11, in <module>
    print(cut_two_before('We all live in a Yellow Submarine', 'Red'))
  File "snippets/exceptions_vs_err_flags.py", line 5, in cut_two_before
    pos = input_string.index(substring)
ValueError: substring not found
\end{Verbatim}
\end{frame}


\begin{frame}
  \frametitle{Enter exceptions}
  \begin{figure}
    \includegraphics[width=0.3\textwidth]{avvocato.jpeg}
    \caption{\footnotesize'Exception, Your Honor!'}
  \end{figure}
  \small
  \begin{itemize}
    \item An exception is an object that can be \alert{raised} (in other languages also \textit{thrown}) by
          a piece of code to signal that something went wrong
    \item When an exception is raised the normal flow of the code is interrupted
    \item The program automatically propagate the exception back in the function hierarchy
          until it found a place where the exception is  \alert{catched} and handled
    \item If the exception is never catched, not even in the main, the program crash \alert{with a specific error message}
    \item Cathcing the exception is done with a \emph{try - except} block
  \end{itemize}
\end{frame}


\begin{frame}
  \frametitle{Try block}
  \begin{Verbatim}[label=\makebox{\url{https://github.com/lucabaldini/cmepda/tree/master/slides/latex/snippets/exceptions\_brief.py}},commandchars=\\\{\}]
\PY{k}{def} \PY{n+nf}{cut\PYZus{}before}\PY{p}{(}\PY{n}{input\PYZus{}string}\PY{p}{,} \PY{n}{substring}\PY{p}{)}\PY{p}{:}
    \PY{k}{try}\PY{p}{:}
        \PY{n}{result} \PY{o}{=} \PY{n}{input\PYZus{}string}\PY{p}{[}\PY{p}{:}\PY{p}{(}\PY{n}{input\PYZus{}string}\PY{o}{.}\PY{n}{index}\PY{p}{(}\PY{n}{substring}\PY{p}{)}\PY{p}{)}\PY{p}{]}
        \PY{k}{print}\PY{p}{(}\PY{l+s+s1}{\PYZsq{}}\PY{l+s+s1}{This line is not executed if an exception is raised in the try block}\PY{l+s+s1}{\PYZsq{}}\PY{p}{)}
        \PY{k}{return} \PY{n}{result}
    \PY{c+c1}{\PYZsh{} Catch the correct exception type with \PYZsq{}except\PYZsq{}}
    \PY{k}{except} \PY{n+ne}{ValueError}\PY{p}{:}
        \PY{k}{print}\PY{p}{(}\PY{l+s+s1}{\PYZsq{}}\PY{l+s+s1}{This line is executed only if a ValueError is raised in the try block}\PY{l+s+s1}{\PYZsq{}}\PY{p}{)}

\PY{n}{cut\PYZus{}before}\PY{p}{(}\PY{l+s+s1}{\PYZsq{}}\PY{l+s+s1}{We all live in a Yellow Submarine}\PY{l+s+s1}{\PYZsq{}}\PY{p}{,} \PY{l+s+s1}{\PYZsq{}}\PY{l+s+s1}{Yellow}\PY{l+s+s1}{\PYZsq{}}\PY{p}{)}
\PY{n}{cut\PYZus{}before}\PY{p}{(}\PY{l+s+s1}{\PYZsq{}}\PY{l+s+s1}{We all live in a Yellow Submarine}\PY{l+s+s1}{\PYZsq{}}\PY{p}{,} \PY{l+s+s1}{\PYZsq{}}\PY{l+s+s1}{Red}\PY{l+s+s1}{\PYZsq{}}\PY{p}{)}

[Output]
This line is not executed if an exception is raised in the try block
This line is executed only if a ValueError is raised in the try block
\end{Verbatim}
\end{frame}


\begin{frame}
  \frametitle{\emph{else}, \emph{finally}}
  \begin{itemize}
    \item There are two more optional statements in a try-block:
    \medskip
    \begin{itemize}
      \item \emph{else}: executed only if no exception is raised in the try block
      \medskip
      \item \emph{finally}: executed no matter what
      \medskip
    \end{itemize}
    \item \emph{finally} is executed even if there is a return statement in the try
          block
    \item can be used to release important resources (e.g. closing a
          file, or a connection)
  \end{itemize}
\end{frame}


\begin{frame}
  \frametitle{Using \emph{else} and \emph{finally}}
  \begin{Verbatim}[label=\makebox{\url{https://github.com/lucabaldini/cmepda/tree/master/slides/latex/snippets/exceptions.py}},commandchars=\\\{\}]
\PY{k}{def} \PY{n+nf}{cut\PYZus{}before}\PY{p}{(}\PY{n}{input\PYZus{}string}\PY{p}{,} \PY{n}{substring}\PY{p}{)}\PY{p}{:}
    \PY{k}{try}\PY{p}{:}
        \PY{n}{result} \PY{o}{=} \PY{n}{input\PYZus{}string}\PY{p}{[}\PY{p}{:}\PY{p}{(}\PY{n}{input\PYZus{}string}\PY{o}{.}\PY{n}{index}\PY{p}{(}\PY{n}{substring}\PY{p}{)}\PY{p}{)}\PY{p}{]}
        \PY{k}{print}\PY{p}{(}\PY{l+s+s1}{\PYZsq{}}\PY{l+s+s1}{This line is not executed if an exception is raised in the try block}\PY{l+s+s1}{\PYZsq{}}\PY{p}{)}
    \PY{k}{except} \PY{n+ne}{ValueError}\PY{p}{:}
        \PY{k}{print}\PY{p}{(}\PY{l+s+s1}{\PYZsq{}}\PY{l+s+s1}{This line is executed only if a ValueError is raised in the try block}\PY{l+s+s1}{\PYZsq{}}\PY{p}{)}
    \PY{k}{else}\PY{p}{:} \PY{c+c1}{\PYZsh{} optional!}
      \PY{k}{print}\PY{p}{(}\PY{l+s+s1}{\PYZsq{}}\PY{l+s+s1}{This line is executed only if no exception is raised in the try block}\PY{l+s+s1}{\PYZsq{}}\PY{p}{)}
      \PY{k}{return} \PY{n}{result}
    \PY{k}{finally}\PY{p}{:} \PY{c+c1}{\PYZsh{} optional!}
      \PY{k}{print}\PY{p}{(}\PY{l+s+s1}{\PYZsq{}}\PY{l+s+s1}{This line is always executed}\PY{l+s+s1}{\PYZsq{}}\PY{p}{)}

\PY{n}{cut\PYZus{}before}\PY{p}{(}\PY{l+s+s1}{\PYZsq{}}\PY{l+s+s1}{We all live in a Yellow Submarine}\PY{l+s+s1}{\PYZsq{}}\PY{p}{,} \PY{l+s+s1}{\PYZsq{}}\PY{l+s+s1}{Yellow}\PY{l+s+s1}{\PYZsq{}}\PY{p}{)}
\PY{n}{cut\PYZus{}before}\PY{p}{(}\PY{l+s+s1}{\PYZsq{}}\PY{l+s+s1}{We all live in a Yellow Submarine}\PY{l+s+s1}{\PYZsq{}}\PY{p}{,} \PY{l+s+s1}{\PYZsq{}}\PY{l+s+s1}{Red}\PY{l+s+s1}{\PYZsq{}}\PY{p}{)}

[Output]
This line is not executed if an exception is raised in the try block
This line is executed only if no exception is raised in the try block
This line is always executed
This line is executed only if a ValueError is raised in the try block
This line is always executed
\end{Verbatim}
\end{frame}


\begin{frame}
  \frametitle{The beauty of exceptions}
  \begin{itemize}
    \item If that was all, exceptions would only be moderately useful
    \item The real bargain is that you can send back information together with the exception
    \item In fact you \textit{are sending a full object}: the excetpion iteslf. Surprised?
    \item Inside the exception you can report all kind of data useful to reconstruct the exact error,
          which can be used by the caller for debug or to produce meaningful error messages
    \item You can also select which exceptions you catch, leaving the others propagate up
    \item Python provides a rich hierarchy of exception classes, which you can further customize
          (if you want) by deriving your own subclasses
  \end{itemize}
\end{frame}


\begin{frame}
  \frametitle{The family tree of Python exceptions}
  \centering\includegraphics[height=0.8\textheight]{python_exceptions}
\end{frame}


\begin{frame}
  \frametitle{Catching specific exceptions}
  \begin{Verbatim}[label=\makebox{\url{https://github.com/lucabaldini/cmepda/tree/master/slides/latex/snippets/try\_block.py}},commandchars=\\\{\}]
\PY{k}{try}\PY{p}{:}
    \PY{k}{with} \PY{n+nb}{open} \PY{p}{(}\PY{l+s+s1}{\PYZsq{}}\PY{l+s+s1}{i\PYZus{}do\PYZus{}not\PYZus{}exist.txt}\PY{l+s+s1}{\PYZsq{}}\PY{p}{)} \PY{k}{as} \PY{n}{lab\PYZus{}data\PYZus{}file}\PY{p}{:}
        \PY{l+s+sd}{\PYZdq{}\PYZdq{}\PYZdq{} Do some process here...}
\PY{l+s+sd}{        ...}
\PY{l+s+sd}{        ...}
\PY{l+s+sd}{        \PYZdq{}\PYZdq{}\PYZdq{}}
\PY{k}{except} \PY{n}{FileNotFoundError} \PY{k}{as} \PY{n}{e}\PY{p}{:} \PY{c+c1}{\PYZsh{} we assign a name to the the exception }
    \PY{k}{print}\PY{p}{(}\PY{n}{e}\PY{p}{)}


\PY{c+c1}{\PYZsh{} We can be less specific by catching a parent exception}
\PY{k}{try}\PY{p}{:}
    \PY{k}{with} \PY{n+nb}{open} \PY{p}{(}\PY{l+s+s1}{\PYZsq{}}\PY{l+s+s1}{i\PYZus{}do\PYZus{}not\PYZus{}exist.txt}\PY{l+s+s1}{\PYZsq{}}\PY{p}{)} \PY{k}{as} \PY{n}{lab\PYZus{}data\PYZus{}file}\PY{p}{:}
        \PY{l+s+sd}{\PYZdq{}\PYZdq{}\PYZdq{} Do some process here...}
\PY{l+s+sd}{        ...}
\PY{l+s+sd}{        ...}
\PY{l+s+sd}{        \PYZdq{}\PYZdq{}\PYZdq{}}
\PY{k}{except} \PY{n+ne}{OSError} \PY{k}{as} \PY{n}{e}\PY{p}{:} \PY{c+c1}{\PYZsh{} OSError is a parent class of FileNotFoundError}
    \PY{k}{print}\PY{p}{(}\PY{n}{e}\PY{p}{)}

\PY{c+c1}{\PYZsh{} catching Exception will catch almost everything!}

[Output]
[Errno 2] No such file or directory: 'i_do_not_exist.txt'
[Errno 2] No such file or directory: 'i_do_not_exist.txt'
\end{Verbatim}
\end{frame}


\begin{frame}
  \frametitle{Exception caveats}
  \begin{itemize}
    \item Warning: catching \emph{Exception}, will also catch \emph{SyntaxError}
          and \emph{NameError}
    \medskip
    \item This mean that the code will 'run' even if there is a typo in it!
    \medskip
    \item Bottom line: \alert{you should never catch generically for \emph{Exception}},
          always be more specific
    \medskip
    \item Even worse, you should never catch for \emph{BaseException} as
          that would even prevent the user for from aborting the execution with a
          \emph{KeyboardInterrupt} (e.g. Ctrl-C)
    \medskip
    \item Unless that is what you need, of course
  \end{itemize}
\end{frame}


\begin{frame}
  \frametitle{There is no check - only try}
  \begin{itemize}
    \item In Python exceptions are the default methods for handling failures
    \smallskip
    \item Many functions raise an exception when something goes wrong
    \smallskip
    \item The common approach is: do not chech the input beforehand. Use it and
          be ready to catch exceptions if any.
    \smallskip
    \item \textit{Easier to ask for forgiveness than permission.} 
  \end{itemize}
\end{frame}


\begin{frame}
  \frametitle{Easier to ask for forgiveness}
  \begin{Verbatim}[label=\makebox{\url{https://github.com/lucabaldini/cmepda/tree/master/slides/latex/snippets/dont\_ask\_permission.py}},commandchars=\\\{\}]
\PY{k+kn}{import} \PY{n+nn}{os}

\PY{n}{file\PYZus{}path} \PY{o}{=} \PY{l+s+s1}{\PYZsq{}}\PY{l+s+s1}{i\PYZus{}do\PYZus{}not\PYZus{}exists.txt}\PY{l+s+s1}{\PYZsq{}}

\PY{c+c1}{\PYZsh{} Defensive version}
\PY{k}{if} \PY{n}{os}\PY{o}{.}\PY{n}{path}\PY{o}{.}\PY{n}{exists}\PY{p}{(}\PY{n}{file\PYZus{}path}\PY{p}{)}\PY{p}{:}
    \PY{c+c1}{\PYZsh{} What if the file is deleted between these two lines? (by another process)}
    \PY{c+c1}{\PYZsh{} What if the file exists but you don\PYZsq{}t have permission to open it?}
    \PY{n}{data\PYZus{}file} \PY{o}{=} \PY{n+nb}{open}\PY{p}{(}\PY{n}{file\PYZus{}path}\PY{p}{)} 
\PY{k}{else}\PY{p}{:}
    \PY{c+c1}{\PYZsh{} Do something}
    \PY{k}{print}\PY{p}{(}\PY{l+s+s1}{\PYZsq{}}\PY{l+s+s1}{Oops \PYZhy{} file }\PY{l+s+se}{\PYZbs{}\PYZsq{}}\PY{l+s+s1}{\PYZob{}\PYZcb{}}\PY{l+s+se}{\PYZbs{}\PYZsq{}}\PY{l+s+s1}{ does not exist}\PY{l+s+s1}{\PYZsq{}}\PY{o}{.}\PY{n}{format}\PY{p}{(}\PY{n}{file\PYZus{}path}\PY{p}{)}\PY{p}{)}

\PY{c+c1}{\PYZsh{} Pythonic way \PYZhy{} you should prefer this one!}
\PY{k}{try}\PY{p}{:}
    \PY{n}{data\PYZus{}file} \PY{o}{=} \PY{n+nb}{open}\PY{p}{(}\PY{n}{file\PYZus{}path}\PY{p}{)}
\PY{k}{except} \PY{n+ne}{OSError} \PY{k}{as} \PY{n}{e}\PY{p}{:} \PY{c+c1}{\PYZsh{} Cover more problems than FileNotFoundError}
    \PY{k}{print}\PY{p}{(}\PY{l+s+s1}{\PYZsq{}}\PY{l+s+s1}{Oops \PYZhy{} cannot read the file!}\PY{l+s+se}{\PYZbs{}n}\PY{l+s+s1}{\PYZob{}\PYZcb{}}\PY{l+s+s1}{\PYZsq{}}\PY{o}{.}\PY{n}{format}\PY{p}{(}\PY{n}{e}\PY{p}{)}\PY{p}{)}

[Output]
Oops - file 'i_do_not_exists.txt' does not exist
Oops - cannot read the file!
[Errno 2] No such file or directory: 'i_do_not_exists.txt'
\end{Verbatim}
\end{frame}


\begin{frame}
  \frametitle{The traceback}
  \begin{itemize}
    \item A \alert{stack traceback}, or simply traceback, is a list of the
          function calls at a specific point in the code
    \smallskip
    \item An exception always stores the traceback at the failure point, since
          it contains crucial debug information 
    \smallskip
    \item In Python a traceback is read from the bottom up:
    \begin{itemize}
    \smallskip
    \item The last line of the traceback is the error message line, which
          contains the name of the exception that was raised
    \smallskip
    \item Further up are the various function calls from most
          recent to least recent
    \end{itemize}
    \medskip
    \item When you catch an exception, the content of the traceback can be
          displayed using either the \emph{logging} module or the
          \emph{traceback} module
  \end{itemize}
\end{frame}


\begin{frame}
  \frametitle{Traceback report with \emph{logging}}
  \begin{Verbatim}[label=\makebox{\url{https://github.com/lucabaldini/cmepda/tree/master/slides/latex/snippets/traceback\_logging.py}},commandchars=\\\{\}]
\PY{l+s+sd}{\PYZdq{}\PYZdq{}\PYZdq{} Showing the tracebak with logging\PYZdq{}\PYZdq{}\PYZdq{}}
\PY{k+kn}{import} \PY{n+nn}{logging}
\PY{n}{logger} \PY{o}{=} \PY{n}{logging}\PY{o}{.}\PY{n}{Logger}\PY{p}{(}\PY{l+s+s1}{\PYZsq{}}\PY{l+s+s1}{demo}\PY{l+s+s1}{\PYZsq{}}\PY{p}{)}

\PY{k}{try}\PY{p}{:}
    \PY{k}{with} \PY{n+nb}{open} \PY{p}{(}\PY{l+s+s1}{\PYZsq{}}\PY{l+s+s1}{i\PYZus{}do\PYZus{}not\PYZus{}exist.txt}\PY{l+s+s1}{\PYZsq{}}\PY{p}{)} \PY{k}{as} \PY{n}{myfile}\PY{p}{:}
        \PY{k}{pass}
\PY{k}{except} \PY{n}{FileNotFoundError} \PY{k}{as} \PY{n}{e}\PY{p}{:}
    \PY{n}{logger}\PY{o}{.}\PY{n}{exception}\PY{p}{(}\PY{n}{e}\PY{p}{)}

[Output]
[Errno 2] No such file or directory: 'i_do_not_exist.txt'
Traceback (most recent call last):
  File "snippets/traceback_logging.py", line 6, in <module>
    with open ('i_do_not_exist.txt') as myfile:
FileNotFoundError: [Errno 2] No such file or directory: 'i_do_not_exist.txt'
\end{Verbatim}
\end{frame}


\begin{frame}
  \frametitle{Traceback report with \emph{traceback}}
  \begin{Verbatim}[label=\makebox{\url{https://github.com/lucabaldini/cmepda/tree/master/slides/latex/snippets/traceback\_tb.py}},commandchars=\\\{\}]
\PY{l+s+sd}{\PYZdq{}\PYZdq{}\PYZdq{} Printing the traceback with the \PYZsq{}traceback\PYZsq{} module \PYZdq{}\PYZdq{}\PYZdq{}}
\PY{k+kn}{import} \PY{n+nn}{traceback}
\PY{k+kn}{import} \PY{n+nn}{sys}

\PY{k}{try}\PY{p}{:}
    \PY{k}{with} \PY{n+nb}{open} \PY{p}{(}\PY{l+s+s1}{\PYZsq{}}\PY{l+s+s1}{i\PYZus{}do\PYZus{}not\PYZus{}exist.txt}\PY{l+s+s1}{\PYZsq{}}\PY{p}{)} \PY{k}{as} \PY{n}{myfile}\PY{p}{:}
        \PY{k}{pass}
\PY{k}{except} \PY{n}{FileNotFoundError} \PY{k}{as} \PY{n}{e}\PY{p}{:}
    \PY{c+c1}{\PYZsh{} The tracebak is stored as an attribute of the exception}
    \PY{c+c1}{\PYZsh{} By default print\PYZus{}tb() output is redirected to the stderr, but you can }
    \PY{c+c1}{\PYZsh{} change that by setting the \PYZsq{}file\PYZsq{} argument}
    \PY{n}{traceback}\PY{o}{.}\PY{n}{print\PYZus{}tb}\PY{p}{(}\PY{n}{e}\PY{o}{.}\PY{n}{\PYZus{}\PYZus{}traceback\PYZus{}\PYZus{}}\PY{p}{,} \PY{n+nb}{file}\PY{o}{=}\PY{n}{sys}\PY{o}{.}\PY{n}{stdout}\PY{p}{)}

[Output]
  File "snippets/traceback_tb.py", line 6, in <module>
    with open ('i_do_not_exist.txt') as myfile:
\end{Verbatim}
\end{frame}


\begin{frame}
  \frametitle{Exception logging with \emph{traceback}}
  \begin{Verbatim}[label=\makebox{\url{https://github.com/lucabaldini/cmepda/tree/master/slides/latex/snippets/traceback\_exception.py}},commandchars=\\\{\}]
\PY{l+s+sd}{\PYZdq{}\PYZdq{}\PYZdq{} Printing the exception with the \PYZsq{}traceback\PYZsq{} module \PYZdq{}\PYZdq{}\PYZdq{}}
\PY{k+kn}{import} \PY{n+nn}{traceback}
\PY{k+kn}{import} \PY{n+nn}{os}

\PY{k}{try}\PY{p}{:}
    \PY{k}{with} \PY{n+nb}{open} \PY{p}{(}\PY{l+s+s1}{\PYZsq{}}\PY{l+s+s1}{i\PYZus{}do\PYZus{}not\PYZus{}exist.txt}\PY{l+s+s1}{\PYZsq{}}\PY{p}{)} \PY{k}{as} \PY{n}{myfile}\PY{p}{:}
        \PY{k}{pass}
\PY{k}{except} \PY{n}{FileNotFoundError} \PY{k}{as} \PY{n}{e}\PY{p}{:}
    \PY{c+c1}{\PYZsh{} Log exception to file}
    \PY{k}{with} \PY{n+nb}{open} \PY{p}{(}\PY{l+s+s1}{\PYZsq{}}\PY{l+s+s1}{traceback.log}\PY{l+s+s1}{\PYZsq{}}\PY{p}{,} \PY{l+s+s1}{\PYZsq{}}\PY{l+s+s1}{w}\PY{l+s+s1}{\PYZsq{}}\PY{p}{)} \PY{k}{as} \PY{n}{logfile}\PY{p}{:}
        \PY{c+c1}{\PYZsh{} This will print the full exception}
        \PY{n}{traceback}\PY{o}{.}\PY{n}{print\PYZus{}exception}\PY{p}{(}\PY{n}{e}\PY{o}{.}\PY{n+nv+vm}{\PYZus{}\PYZus{}class\PYZus{}\PYZus{}}\PY{p}{,} \PY{n}{e}\PY{p}{,} \PY{n}{e}\PY{o}{.}\PY{n}{\PYZus{}\PYZus{}traceback\PYZus{}\PYZus{}}\PY{p}{,} \PY{n+nb}{file}\PY{o}{=}\PY{n}{logfile}\PY{p}{)}
        \PY{c+c1}{\PYZsh{} Since Python 3.10 this was simplified into:}
        \PY{c+c1}{\PYZsh{} traceback.print\PYZus{}exception(e)}
    \PY{n}{os}\PY{o}{.}\PY{n}{system}\PY{p}{(}\PY{l+s+s1}{\PYZsq{}}\PY{l+s+s1}{cat traceback.log}\PY{l+s+s1}{\PYZsq{}}\PY{p}{)}

[Output]
Traceback (most recent call last):
  File "snippets/traceback_exception.py", line 6, in <module>
    with open ('i_do_not_exist.txt') as myfile:
FileNotFoundError: [Errno 2] No such file or directory: 'i_do_not_exist.txt'
\end{Verbatim}
\end{frame}


\begin{frame}
  \frametitle{Raising exceptions}
  \begin{itemize}
    \item Up to now we have been dealing with exceptions generated by Python
          functions
    \medskip
    \item What about raising exceptions ourselves?
  \end{itemize}
\end{frame}


\begin{frame}
  \frametitle{Raising exceptions}
  \begin{Verbatim}[label=\makebox{\url{https://github.com/lucabaldini/cmepda/tree/master/slides/latex/snippets/raising.py}},commandchars=\\\{\}]
\PY{k}{def} \PY{n+nf}{raising\PYZus{}function}\PY{p}{(}\PY{p}{)}\PY{p}{:}
    \PY{c+c1}{\PYZsh{} You can pass an useful message to the exceptions you raise}
    \PY{k}{raise} \PY{n+ne}{RuntimeError}\PY{p}{(}\PY{l+s+s1}{\PYZsq{}}\PY{l+s+s1}{this is a useful debug message}\PY{l+s+s1}{\PYZsq{}}\PY{p}{)} 

\PY{k}{try}\PY{p}{:}
    \PY{n}{raising\PYZus{}function}\PY{p}{(}\PY{p}{)}
\PY{k}{except} \PY{n+ne}{RuntimeError} \PY{k}{as} \PY{n}{e}\PY{p}{:}
    \PY{c+c1}{\PYZsh{} The message can be retrieved by printing the exception}
    \PY{k}{print}\PY{p}{(}\PY{n}{e}\PY{p}{)}

[Output]
this is a useful debug message
\end{Verbatim}
\end{frame}


\begin{frame}
  \frametitle{Custom exceptions}
  \begin{itemize}
    \item Beside the built-in exceptions provided by Python, you can add your
          own custom exceptions by inheriting from the \emph{Exception} class
    \medskip
    \item This serves two purposes:
    \begin{itemize}
      \item Make the exception handling code more specific, and hence more
            readible
      \item Allows you to pass additional data with your exception - in the
            form of attributes of the class - which can be used for debug
            or any other purpose
    \end{itemize}
  \end{itemize}
\end{frame}


\begin{frame}
  \frametitle{Custom exceptions}
  \begin{Verbatim}[label=\makebox{\url{https://github.com/lucabaldini/cmepda/tree/master/slides/latex/snippets/custom\_exceptions.py}},commandchars=\\\{\}]
\PY{k}{class} \PY{n+nc}{SimpleCustomError}\PY{p}{(}\PY{n+ne}{Exception}\PY{p}{)}\PY{p}{:}  
    \PY{k}{pass} \PY{c+c1}{\PYZsh{} Yeah that\PYZsq{}s it}

\PY{k}{raise} \PY{n}{SimpleCustomError}\PY{p}{(}\PY{l+s+s1}{\PYZsq{}}\PY{l+s+s1}{simple error}\PY{l+s+s1}{\PYZsq{}}\PY{p}{)}

[Output]
Traceback (most recent call last):
  File "snippets/custom_exceptions.py", line 4, in <module>
    raise SimpleCustomError('simple error')
__main__.SimpleCustomError: simple error
\end{Verbatim}
\end{frame}


\begin{frame}
  \frametitle{Custom exceptions}
  \begin{Verbatim}[label=\makebox{\url{https://github.com/lucabaldini/cmepda/tree/master/slides/latex/snippets/custom\_exceptions\_2.py}},commandchars=\\\{\}]
\PY{k}{class} \PY{n+nc}{ValueTooLargeError}\PY{p}{(}\PY{n+ne}{ValueError}\PY{p}{)}\PY{p}{:}
    \PY{k}{def} \PY{n+nf+fm}{\PYZus{}\PYZus{}init\PYZus{}\PYZus{}}\PY{p}{(}\PY{n+nb+bp}{self}\PY{p}{,} \PY{n}{value}\PY{p}{)}\PY{p}{:}
        \PY{n+nb+bp}{self}\PY{o}{.}\PY{n}{value} \PY{o}{=} \PY{n}{value}
        \PY{n+nb}{super}\PY{p}{(}\PY{p}{)}\PY{o}{.}\PY{n+nf+fm}{\PYZus{}\PYZus{}init\PYZus{}\PYZus{}}\PY{p}{(}\PY{l+s+sa}{f}\PY{l+s+s1}{\PYZsq{}}\PY{l+s+si}{\PYZob{}}\PY{n+nb+bp}{self}\PY{o}{.}\PY{n+nv+vm}{\PYZus{}\PYZus{}class\PYZus{}\PYZus{}}\PY{o}{.}\PY{n+nv+vm}{\PYZus{}\PYZus{}name\PYZus{}\PYZus{}}\PY{l+s+si}{\PYZcb{}}\PY{l+s+s1}{: }\PY{l+s+si}{\PYZob{}}\PY{n+nb+bp}{self}\PY{o}{.}\PY{n}{value}\PY{l+s+si}{\PYZcb{}}\PY{l+s+s1}{ is too large}\PY{l+s+s1}{\PYZsq{}}\PY{p}{)}

\PY{n}{value} \PY{o}{=} \PY{l+m+mi}{100}
\PY{k}{try}\PY{p}{:}
  \PY{k}{if} \PY{n}{value} \PY{o}{\PYZgt{}} \PY{l+m+mi}{10}\PY{p}{:}
      \PY{k}{raise} \PY{n}{ValueTooLargeError}\PY{p}{(}\PY{n}{value}\PY{p}{)}
\PY{k}{except} \PY{n+ne}{ValueError} \PY{k}{as} \PY{n}{e}\PY{p}{:}
    \PY{n+nb}{print}\PY{p}{(}\PY{n}{e}\PY{p}{)}

[Output]
ValueTooLargeError: 100 is too large
\end{Verbatim}
\end{frame}


\begin{frame}
  \frametitle{Where to catch exceptions?}
  \begin{itemize}
    \item Differently from error flags, which need to be checked as early as
          possible, you are not in a rush with exceptions
    \item Remeber: your goal is to provide the user a meaningful error message and
          useful debug information.
    \item You should catch an exception only when you have enough context to
          do that - which sometimes means waiting a few levels in the hierarchy!
  \end{itemize}
\end{frame}


\begin{frame}
  \frametitle{When to catch?}
  \centering
  \includegraphics[width=0.8\textwidth]{lab_file.png}
\end{frame}


\begin{frame}
  \frametitle{When to catch?}
  \begin{Verbatim}[label=\makebox{\url{https://github.com/lucabaldini/cmepda/tree/master/slides/latex/snippets/when\_to\_catch.py}},commandchars=\\\{\}]
\PY{k}{def} \PY{n+nf}{parse\PYZus{}line}\PY{p}{(}\PY{n}{line}\PY{p}{)}\PY{p}{:}
    \PY{l+s+sd}{\PYZdq{}\PYZdq{}\PYZdq{} Parse a line of the file and return the values as float\PYZdq{}\PYZdq{}\PYZdq{}}
    \PY{n}{values} \PY{o}{=} \PY{n}{line}\PY{o}{.}\PY{n}{strip}\PY{p}{(}\PY{l+s+s1}{\PYZsq{}}\PY{l+s+se}{\PYZbs{}n}\PY{l+s+s1}{\PYZsq{}}\PY{p}{)}\PY{o}{.}\PY{n}{split}\PY{p}{(}\PY{l+s+s1}{\PYZsq{}}\PY{l+s+s1}{ }\PY{l+s+s1}{\PYZsq{}}\PY{p}{)}
    \PY{c+c1}{\PYZsh{} the following two lines may generate exceptions if they fails!}
    \PY{n}{time} \PY{o}{=} \PY{n+nb}{float}\PY{p}{(}\PY{n}{values}\PY{p}{[}\PY{l+m+mi}{0}\PY{p}{]}\PY{p}{)}
    \PY{n}{tension} \PY{o}{=} \PY{n+nb}{float}\PY{p}{(}\PY{n}{values}\PY{p}{[}\PY{l+m+mi}{1}\PY{p}{]}\PY{p}{)}
    \PY{k}{return} \PY{n}{time}\PY{p}{,} \PY{n}{tension}

\PY{k}{with} \PY{n+nb}{open}\PY{p}{(}\PY{l+s+s1}{\PYZsq{}}\PY{l+s+s1}{fake\PYZus{}measurements.txt}\PY{l+s+s1}{\PYZsq{}}\PY{p}{)} \PY{k}{as} \PY{n}{lab\PYZus{}data\PYZus{}file}\PY{p}{:}
    \PY{k}{for} \PY{n}{line} \PY{o+ow}{in} \PY{n}{lab\PYZus{}data\PYZus{}file}\PY{p}{:}
        \PY{k}{if} \PY{o+ow}{not} \PY{n}{line}\PY{o}{.}\PY{n}{startswith}\PY{p}{(}\PY{l+s+s1}{\PYZsq{}}\PY{l+s+s1}{\PYZsh{}}\PY{l+s+s1}{\PYZsq{}}\PY{p}{)}\PY{p}{:} \PY{c+c1}{\PYZsh{} skip comments}
            \PY{n}{time}\PY{p}{,} \PY{n}{tension} \PY{o}{=} \PY{n}{parse\PYZus{}line}\PY{p}{(}\PY{n}{line}\PY{p}{)}
            \PY{k}{print}\PY{p}{(}\PY{n}{time}\PY{p}{,} \PY{n}{tension}\PY{p}{)}

[Output]
0.1 15.2
0.2 12.4
Traceback (most recent call last):
  File "snippets/when_to_catch.py", line 12, in <module>
    time, tension = parse_line(line)
  File "snippets/when_to_catch.py", line 6, in parse_line
    tension = float(values[1])
ValueError: could not convert string to float: 'pippo'
\end{Verbatim}
\end{frame}


\begin{frame}
  \frametitle{Catch too early}
  \begin{Verbatim}[label=\makebox{\url{https://github.com/lucabaldini/cmepda/tree/master/slides/latex/snippets/when\_to\_catch\_1.py}},commandchars=\\\{\}]
\PY{k}{def} \PY{n+nf}{parse\PYZus{}line}\PY{p}{(}\PY{n}{line}\PY{p}{)}\PY{p}{:}
    \PY{l+s+sd}{\PYZdq{}\PYZdq{}\PYZdq{} Parse a line of the file and return the values as float\PYZdq{}\PYZdq{}\PYZdq{}}
    \PY{n}{values} \PY{o}{=} \PY{n}{line}\PY{o}{.}\PY{n}{strip}\PY{p}{(}\PY{l+s+s1}{\PYZsq{}}\PY{l+s+se}{\PYZbs{}n}\PY{l+s+s1}{\PYZsq{}}\PY{p}{)}\PY{o}{.}\PY{n}{split}\PY{p}{(}\PY{l+s+s1}{\PYZsq{}}\PY{l+s+s1}{ }\PY{l+s+s1}{\PYZsq{}}\PY{p}{)}
    \PY{k}{try}\PY{p}{:}
        \PY{n}{time} \PY{o}{=} \PY{n+nb}{float}\PY{p}{(}\PY{n}{values}\PY{p}{[}\PY{l+m+mi}{0}\PY{p}{]}\PY{p}{)}
        \PY{n}{voltage} \PY{o}{=} \PY{n+nb}{float}\PY{p}{(}\PY{n}{values}\PY{p}{[}\PY{l+m+mi}{1}\PY{p}{]}\PY{p}{)}
    \PY{k}{except} \PY{n+ne}{ValueError} \PY{k}{as} \PY{n}{e}\PY{p}{:}
        \PY{n+nb}{print}\PY{p}{(}\PY{n}{e}\PY{p}{)} \PY{c+c1}{\PYZsh{} This is not useful \PYZhy{} which line of the file has the error?}
        \PY{k}{return} \PY{k+kc}{None} \PY{c+c1}{\PYZsh{} We can\PYZsq{}t really return something meaningful}
    \PY{k}{return} \PY{n}{time}\PY{p}{,} \PY{n}{voltage}

\PY{k}{with} \PY{n+nb}{open}\PY{p}{(}\PY{l+s+s1}{\PYZsq{}}\PY{l+s+s1}{snippets/data/fake\PYZus{}measurements.txt}\PY{l+s+s1}{\PYZsq{}}\PY{p}{)} \PY{k}{as} \PY{n}{lab\PYZus{}data\PYZus{}file}\PY{p}{:}
    \PY{k}{for} \PY{n}{line} \PY{o+ow}{in} \PY{n}{lab\PYZus{}data\PYZus{}file}\PY{p}{:}
        \PY{k}{if} \PY{o+ow}{not} \PY{n}{line}\PY{o}{.}\PY{n}{startswith}\PY{p}{(}\PY{l+s+s1}{\PYZsq{}}\PY{l+s+s1}{\PYZsh{}}\PY{l+s+s1}{\PYZsq{}}\PY{p}{)}\PY{p}{:} \PY{c+c1}{\PYZsh{} skip comments}
            \PY{n}{time}\PY{p}{,} \PY{n}{voltage} \PY{o}{=} \PY{n}{parse\PYZus{}line}\PY{p}{(}\PY{n}{line}\PY{p}{)}
            \PY{n+nb}{print}\PY{p}{(}\PY{n}{time}\PY{p}{,} \PY{n}{voltage}\PY{p}{)} \PY{c+c1}{\PYZsh{} This line still crash badly!}

[Output]
0.1 15.2
0.2 12.4
could not convert string to float: 'pippo'
Traceback (most recent call last):
  File "/home/data/work/teaching/cmepda/slides/latex/snippets/when_to_catch_1.py", line 15, in <module>
    time, voltage = parse_line(line)
TypeError: cannot unpack non-iterable NoneType object
\end{Verbatim}
\end{frame}


\begin{frame}
  \frametitle{Catch when needed}
  \begin{Verbatim}[label=\makebox{\url{https://github.com/lucabaldini/cmepda/tree/master/slides/latex/snippets/when\_to\_catch\_2.py}},commandchars=\\\{\}]
\PY{k}{def} \PY{n+nf}{parse\PYZus{}line}\PY{p}{(}\PY{n}{line}\PY{p}{)}\PY{p}{:}
    \PY{l+s+sd}{\PYZdq{}\PYZdq{}\PYZdq{} Parse a line of the file and return the values as float\PYZdq{}\PYZdq{}\PYZdq{}}
    \PY{n}{values} \PY{o}{=} \PY{n}{line}\PY{o}{.}\PY{n}{strip}\PY{p}{(}\PY{l+s+s1}{\PYZsq{}}\PY{l+s+se}{\PYZbs{}n}\PY{l+s+s1}{\PYZsq{}}\PY{p}{)}\PY{o}{.}\PY{n}{split}\PY{p}{(}\PY{l+s+s1}{\PYZsq{}}\PY{l+s+s1}{ }\PY{l+s+s1}{\PYZsq{}}\PY{p}{)}
    \PY{n}{time} \PY{o}{=} \PY{n+nb}{float}\PY{p}{(}\PY{n}{values}\PY{p}{[}\PY{l+m+mi}{0}\PY{p}{]}\PY{p}{)}
    \PY{n}{voltage} \PY{o}{=} \PY{n+nb}{float}\PY{p}{(}\PY{n}{values}\PY{p}{[}\PY{l+m+mi}{1}\PY{p}{]}\PY{p}{)}
    \PY{k}{return} \PY{n}{time}\PY{p}{,} \PY{n}{voltage}

\PY{k}{with} \PY{n+nb}{open}\PY{p}{(}\PY{l+s+s1}{\PYZsq{}}\PY{l+s+s1}{snippets/data/fake\PYZus{}measurements.txt}\PY{l+s+s1}{\PYZsq{}}\PY{p}{)} \PY{k}{as} \PY{n}{lab\PYZus{}data\PYZus{}file}\PY{p}{:}
    \PY{k}{for} \PY{n}{line\PYZus{}number}\PY{p}{,} \PY{n}{line} \PY{o+ow}{in} \PY{n+nb}{enumerate}\PY{p}{(}\PY{n}{lab\PYZus{}data\PYZus{}file}\PY{p}{)}\PY{p}{:} \PY{c+c1}{\PYZsh{} get the line number}
        \PY{k}{if} \PY{o+ow}{not} \PY{n}{line}\PY{o}{.}\PY{n}{startswith}\PY{p}{(}\PY{l+s+s1}{\PYZsq{}}\PY{l+s+s1}{\PYZsh{}}\PY{l+s+s1}{\PYZsq{}}\PY{p}{)}\PY{p}{:} \PY{c+c1}{\PYZsh{} skip comments}
            \PY{k}{try}\PY{p}{:}
                \PY{n}{time}\PY{p}{,} \PY{n}{voltage} \PY{o}{=} \PY{n}{parse\PYZus{}line}\PY{p}{(}\PY{n}{line}\PY{p}{)}
                \PY{n+nb}{print}\PY{p}{(}\PY{n}{time}\PY{p}{,} \PY{n}{voltage}\PY{p}{)}
            \PY{k}{except} \PY{n+ne}{ValueError} \PY{k}{as} \PY{n}{e}\PY{p}{:}
                \PY{n+nb}{print}\PY{p}{(}\PY{l+s+s1}{\PYZsq{}}\PY{l+s+s1}{Line }\PY{l+s+si}{\PYZob{}\PYZcb{}}\PY{l+s+s1}{ error: }\PY{l+s+si}{\PYZob{}\PYZcb{}}\PY{l+s+s1}{\PYZsq{}}\PY{o}{.}\PY{n}{format}\PY{p}{(}\PY{n}{line\PYZus{}number}\PY{p}{,} \PY{n}{e}\PY{p}{)}\PY{p}{)}

[Output]
0.1 15.2
0.2 12.4
Line 3 error: could not convert string to float: 'pippo'
0.4 13.2
\end{Verbatim}
\end{frame}


\begin{frame}
  \frametitle{}
  \centering \Large Iterators
\end{frame}


\begin{frame}
  \frametitle{Iterators and iterables}
  
  \begin{itemize}
    \item An \emph{iterable} in Python is something that has a \emph{\_\_iter\_\_}
          method, which returns an \alert{iterator}
    \medskip
    \item An \emph{iterator} is an object that implement a \emph{\_\_next\_\_} method
          which is used to retrieve elements one at the time
    \medskip
    \item When there are no more elements to return, the iterator signals that with a specific
          exception: \emph{StopIteration()}
    \medskip
    \item An iterator also implement an \emph{\_\_iter\_\_} method that return\dots itself.
          So an iterator is also technically an iterable%
          \footnote{Only 'technically' because an iterator has no data of its
          own, so you always need a 'real' iterable to actually iterate}%
          ! (But the opposite is not true)
  \end{itemize}
  
\end{frame}


\begin{frame}
  \frametitle{A 'for' loop unpacked}
  \begin{Verbatim}[label=\makebox{\url{https://bitbucket.org/lbaldini/programming/src/tip/snippets/show\_iterator.py}},commandchars=\\\{\}]
\PY{n}{my\PYZus{}list} \PY{o}{=} \PY{p}{[}\PY{l+m+mf}{1.}\PY{p}{,} \PY{l+m+mf}{2.}\PY{p}{,} \PY{l+m+mf}{3.}\PY{p}{]}

\PY{c+c1}{\PYZsh{} For\PYZhy{}loop syntax}
\PY{k}{for} \PY{n}{element} \PY{o+ow}{in} \PY{n}{my\PYZus{}list}\PY{p}{:}
    \PY{k}{print}\PY{p}{(}\PY{n}{element}\PY{p}{)}

\PY{c+c1}{\PYZsh{} This is equivalent (but much less readible and compact)}
\PY{n}{list\PYZus{}iterator} \PY{o}{=} \PY{n+nb}{iter}\PY{p}{(}\PY{n}{my\PYZus{}list}\PY{p}{)}
\PY{k}{while} \PY{n+nb+bp}{True}\PY{p}{:}
    \PY{c+c1}{\PYZsh{} We will study try\PYZhy{}except statements in the advanced module}
    \PY{c+c1}{\PYZsh{} For now you just need to know that the code in the try block is}
    \PY{c+c1}{\PYZsh{} executed at each iteration until the list is over: at taht point the}
    \PY{c+c1}{\PYZsh{} code in the except block is executed instead. In this case we use the }
    \PY{c+c1}{\PYZsh{} except block to just exit from the while loop}
    \PY{k}{try}\PY{p}{:}
        \PY{k}{print}\PY{p}{(}\PY{n+nb}{next}\PY{p}{(}\PY{n}{list\PYZus{}iterator}\PY{p}{)}\PY{p}{)}
    \PY{k}{except} \PY{n+ne}{StopIteration}\PY{p}{:}
        \PY{k}{break}

[Output]
1.0
2.0
3.0
1.0
2.0
3.0
\end{Verbatim}
\end{frame}


\begin{frame}
  \frametitle{A simple iterator}
  \begin{Verbatim}[label=\makebox{\url{https://bitbucket.org/lbaldini/programming/src/tip/snippets/simple\_iterator.py}},commandchars=\\\{\}]
\PY{k}{class} \PY{n+nc}{SimpleIterator}\PY{p}{:}
    \PY{l+s+sd}{\PYZdq{}\PYZdq{}\PYZdq{} Class implementing a super naive iterator\PYZdq{}\PYZdq{}\PYZdq{}}
    
    \PY{k}{def} \PY{n+nf+fm}{\PYZus{}\PYZus{}init\PYZus{}\PYZus{}}\PY{p}{(}\PY{n+nb+bp}{self}\PY{p}{,} \PY{n}{container}\PY{p}{)}\PY{p}{:}
        \PY{n+nb+bp}{self}\PY{o}{.}\PY{n}{\PYZus{}container} \PY{o}{=} \PY{n}{container}
        \PY{n+nb+bp}{self}\PY{o}{.}\PY{n}{index} \PY{o}{=} \PY{l+m+mi}{0}
    
    \PY{k}{def} \PY{n+nf}{\PYZus{}\PYZus{}next\PYZus{}\PYZus{}}\PY{p}{(}\PY{n+nb+bp}{self}\PY{p}{)}\PY{p}{:}
        \PY{k}{try}\PY{p}{:}
            \PY{c+c1}{\PYZsh{} Note: here we are calling the \PYZus{}\PYZus{}getitem\PYZus{}\PYZus{} method of self.\PYZus{}container}
            \PY{n}{item} \PY{o}{=} \PY{n+nb+bp}{self}\PY{o}{.}\PY{n}{\PYZus{}container}\PY{p}{[}\PY{n+nb+bp}{self}\PY{o}{.}\PY{n}{index}\PY{p}{]}
        \PY{k}{except} \PY{n+ne}{IndexError}\PY{p}{:}
            \PY{k}{raise} \PY{n+ne}{StopIteration}
        \PY{n+nb+bp}{self}\PY{o}{.}\PY{n}{index} \PY{o}{+}\PY{o}{=} \PY{l+m+mi}{1}
        \PY{k}{return} \PY{n}{item}
    
    \PY{k}{def} \PY{n+nf+fm}{\PYZus{}\PYZus{}iter\PYZus{}\PYZus{}}\PY{p}{(}\PY{n+nb+bp}{self}\PY{p}{)}\PY{p}{:}
        \PY{k}{return} \PY{n+nb+bp}{self}
        
\PY{k}{class} \PY{n+nc}{SimpleIterable}\PY{p}{:}
    \PY{l+s+sd}{\PYZdq{}\PYZdq{}\PYZdq{} A very basic iterable \PYZdq{}\PYZdq{}\PYZdq{}}
    
    \PY{k}{def} \PY{n+nf+fm}{\PYZus{}\PYZus{}init\PYZus{}\PYZus{}}\PY{p}{(}\PY{n+nb+bp}{self}\PY{p}{,} \PY{o}{*}\PY{n}{elements}\PY{p}{)}\PY{p}{:}
        \PY{c+c1}{\PYZsh{} We use a list to store elements internally.}
        \PY{c+c1}{\PYZsh{} This provide us with the \PYZus{}\PYZus{}getitem\PYZus{}\PYZus{} function}
        \PY{n+nb+bp}{self}\PY{o}{.}\PY{n}{\PYZus{}elements} \PY{o}{=} \PY{n+nb}{list}\PY{p}{(}\PY{n}{elements}\PY{p}{)}
    
    \PY{k}{def} \PY{n+nf+fm}{\PYZus{}\PYZus{}iter\PYZus{}\PYZus{}}\PY{p}{(}\PY{n+nb+bp}{self}\PY{p}{)}\PY{p}{:}
        \PY{k}{return} \PY{n}{SimpleIterator}\PY{p}{(}\PY{n+nb+bp}{self}\PY{o}{.}\PY{n}{\PYZus{}elements}\PY{p}{)}
\end{Verbatim}
\end{frame}


\begin{frame}
  \frametitle{A simple iterator}
  \begin{Verbatim}[label=\makebox{\url{https://github.com/lucabaldini/cmepda/tree/master/slides/latex/snippets/test\_simple\_iterator.py}},commandchars=\\\{\}]
\PY{k+kn}{from} \PY{n+nn}{simple\PYZus{}iterator} \PY{k+kn}{import} \PY{n}{SimpleIterable}
   
\PY{n}{my\PYZus{}iterable} \PY{o}{=} \PY{n}{SimpleIterable}\PY{p}{(}\PY{l+m+mf}{1.}\PY{p}{,} \PY{l+m+mf}{2.}\PY{p}{,} \PY{l+m+mf}{3.}\PY{p}{,} \PY{l+s+s1}{\PYZsq{}}\PY{l+s+s1}{stella}\PY{l+s+s1}{\PYZsq{}}\PY{p}{)}
\PY{k}{for} \PY{n}{element} \PY{o+ow}{in} \PY{n}{my\PYZus{}iterable}\PY{p}{:}
    \PY{k}{print}\PY{p}{(}\PY{n}{element}\PY{p}{)}

[Output]
1.0
2.0
3.0
stella
\end{Verbatim}
\end{frame}


\begin{frame}
  \frametitle{A crazy iterator}
  \begin{Verbatim}[label=\makebox{\url{https://bitbucket.org/lbaldini/programming/src/tip/snippets/crazy\_iterator.py}},commandchars=\\\{\}]
\PY{k+kn}{import} \PY{n+nn}{random}

\PY{k}{class} \PY{n+nc}{CrazyIterator}\PY{p}{:}
    \PY{l+s+sd}{\PYZdq{}\PYZdq{}\PYZdq{} Class implementing a crazy iterator\PYZdq{}\PYZdq{}\PYZdq{}}
    
    \PY{k}{def} \PY{n+nf+fm}{\PYZus{}\PYZus{}init\PYZus{}\PYZus{}}\PY{p}{(}\PY{n+nb+bp}{self}\PY{p}{,} \PY{n}{container}\PY{p}{)}\PY{p}{:}
        \PY{n}{random}\PY{o}{.}\PY{n}{seed}\PY{p}{(}\PY{l+m+mi}{1}\PY{p}{)}
        \PY{n+nb+bp}{self}\PY{o}{.}\PY{n}{\PYZus{}container} \PY{o}{=} \PY{n}{container}
    
    \PY{k}{def} \PY{n+nf}{\PYZus{}\PYZus{}next\PYZus{}\PYZus{}}\PY{p}{(}\PY{n+nb+bp}{self}\PY{p}{)}\PY{p}{:}
        \PY{k}{try}\PY{p}{:}
            \PY{c+c1}{\PYZsh{} We get one possibility out of len(self.\PYZus{}container) to exit}
            \PY{n}{index} \PY{o}{=} \PY{n}{random}\PY{o}{.}\PY{n}{randint}\PY{p}{(}\PY{l+m+mi}{0}\PY{p}{,} \PY{n+nb}{len}\PY{p}{(}\PY{n+nb+bp}{self}\PY{o}{.}\PY{n}{\PYZus{}container}\PY{p}{)}\PY{p}{)}
            \PY{n}{item} \PY{o}{=} \PY{n+nb+bp}{self}\PY{o}{.}\PY{n}{\PYZus{}container}\PY{p}{[}\PY{n}{index}\PY{p}{]}
        \PY{k}{except} \PY{n+ne}{IndexError}\PY{p}{:}
            \PY{k}{raise} \PY{n+ne}{StopIteration}
        \PY{k}{return} \PY{n}{item}
    
    \PY{k}{def} \PY{n+nf+fm}{\PYZus{}\PYZus{}iter\PYZus{}\PYZus{}}\PY{p}{(}\PY{n+nb+bp}{self}\PY{p}{)}\PY{p}{:}
        \PY{k}{return} \PY{n+nb+bp}{self}

\PY{k}{class} \PY{n+nc}{CrazyIterable}\PY{p}{:}
    \PY{l+s+sd}{\PYZdq{}\PYZdq{}\PYZdq{} Similar to a simple iterable, but with a twist... \PYZdq{}\PYZdq{}\PYZdq{}}
    
    \PY{k}{def} \PY{n+nf+fm}{\PYZus{}\PYZus{}init\PYZus{}\PYZus{}}\PY{p}{(}\PY{n+nb+bp}{self}\PY{p}{,} \PY{o}{*}\PY{n}{elements}\PY{p}{)}\PY{p}{:}
        \PY{n+nb+bp}{self}\PY{o}{.}\PY{n}{\PYZus{}elements} \PY{o}{=} \PY{n+nb}{list}\PY{p}{(}\PY{n}{elements}\PY{p}{)}
    
    \PY{k}{def} \PY{n+nf+fm}{\PYZus{}\PYZus{}iter\PYZus{}\PYZus{}}\PY{p}{(}\PY{n+nb+bp}{self}\PY{p}{)}\PY{p}{:}
        \PY{k}{return} \PY{n}{CrazyIterator}\PY{p}{(}\PY{n+nb+bp}{self}\PY{o}{.}\PY{n}{\PYZus{}elements}\PY{p}{)}
\end{Verbatim}
\end{frame}


\begin{frame}
  \frametitle{A crazy iterator}
  \begin{Verbatim}[label=\makebox{\url{https://bitbucket.org/lbaldini/programming/src/tip/snippets/test\_crazy\_iterator.py}},commandchars=\\\{\}]
\PY{k+kn}{from} \PY{n+nn}{crazy\PYZus{}iterator} \PY{k+kn}{import} \PY{n}{CrazyIterable}
   
\PY{n}{my\PYZus{}iterable} \PY{o}{=} \PY{n}{CrazyIterable}\PY{p}{(}\PY{l+s+s1}{\PYZsq{}}\PY{l+s+s1}{A}\PY{l+s+s1}{\PYZsq{}}\PY{p}{,} \PY{l+s+s1}{\PYZsq{}}\PY{l+s+s1}{B}\PY{l+s+s1}{\PYZsq{}}\PY{p}{,} \PY{l+s+s1}{\PYZsq{}}\PY{l+s+s1}{C}\PY{l+s+s1}{\PYZsq{}}\PY{p}{,} \PY{l+s+s1}{\PYZsq{}}\PY{l+s+s1}{D}\PY{l+s+s1}{\PYZsq{}}\PY{p}{,} \PY{l+s+s1}{\PYZsq{}}\PY{l+s+s1}{E}\PY{l+s+s1}{\PYZsq{}}\PY{p}{)}
\PY{k}{for} \PY{n}{element} \PY{o+ow}{in} \PY{n}{my\PYZus{}iterable}\PY{p}{:}
    \PY{k}{print}\PY{p}{(}\PY{n}{element}\PY{p}{)}

[Output]
B
E
A
C
A
D
D
D
\end{Verbatim}
\end{frame}


\begin{frame}
  \frametitle{Python tools for iterables}
  \begin{itemize}
    \item Python provides a number of functions that consume an iterable and return a single value:
    \begin{itemize}
      \item \emph{sum}: Sum all the elements
      \item \emph{all}: Return true if a given condition is true for all the elements
      \item \emph{any}: Return true if a given condition is true for at lest one element
      \item \emph{max}: Return the max
      \item \emph{min}: Return the minimum
      \item \emph{functools.reduce}: Apply a function recursively to pairs of elements
    \end{itemize}
  \end{itemize}
\end{frame}


\begin{frame}
  \frametitle{}
  \centering \Large Generators
\end{frame}


\begin{frame}
  \frametitle{Generators}
  
  \begin{itemize}
    \item We have seen that iterators are useful to iterate over container
    \item However that assumes a containers exists $\rightarrow$ memory usage
    \item \alert{Generators} allow you to loop over sequences of items even when
          they don't exist before - the items are just created \alert{lazily} the
          moment they are required
    \item For example you can write a generator to loops over the Fibonacci
          succession. You can't create the sequence earlier, since it is not
          finite!
    \item Generators are created through either \alert{generator expressions} or
          \alert{generator functions}
    \item In real life most of the time you will simply use pre-made functions
          that return a generator, like \emph{range()} (in Python 3)
    \item Generator can be used to iterate in for loops, just like iterators
  \end{itemize}
  
\end{frame}


\begin{frame}
  \frametitle{Generators first look}
  \begin{Verbatim}[label=\makebox{\url{https://github.com/lucabaldini/cmepda/tree/master/slides/latex/snippets/generators.py}},commandchars=\\\{\}]
\PY{l+s+sd}{\PYZdq{}\PYZdq{}\PYZdq{} range() is a function that returns a generator in Python 3. The list of }
\PY{l+s+sd}{numbers never exists entirely, they are created ine at a time.}
\PY{l+s+sd}{Note: In Python 2 range() does create the full list at the beginning. }
\PY{l+s+sd}{There used to be a xrange() function for lazy generation, which is now }
\PY{l+s+sd}{deprecated in Python 3. \PYZdq{}\PYZdq{}\PYZdq{}}
\PY{k}{for} \PY{n}{i} \PY{o+ow}{in} \PY{n+nb}{range}\PY{p}{(}\PY{l+m+mi}{4}\PY{p}{)}\PY{p}{:}  \PY{c+c1}{\PYZsh{} generators act like iterators in for loop}
    \PY{k}{print}\PY{p}{(}\PY{n}{i}\PY{p}{)}

\PY{n}{data} \PY{o}{=} \PY{p}{[}\PY{l+m+mi}{12}\PY{p}{,} \PY{o}{\PYZhy{}}\PY{l+m+mi}{1}\PY{p}{,} \PY{l+m+mi}{5}\PY{p}{]}
\PY{n}{square\PYZus{}data\PYZus{}generator} \PY{o}{=} \PY{p}{(}\PY{n}{x}\PY{o}{*}\PY{o}{*}\PY{l+m+mi}{2} \PY{k}{for} \PY{n}{x} \PY{o+ow}{in} \PY{n}{data}\PY{p}{)} \PY{c+c1}{\PYZsh{} generator expression!}
\PY{k}{for} \PY{n}{square\PYZus{}datum} \PY{o+ow}{in} \PY{n}{square\PYZus{}data\PYZus{}generator}\PY{p}{:} \PY{c+c1}{\PYZsh{} again, works like an iterator}
    \PY{k}{print}\PY{p}{(}\PY{n}{square\PYZus{}datum}\PY{p}{)}

[Output]
0
1
2
3
144
1
25
\end{Verbatim}
\end{frame}


\begin{frame}
  \frametitle{Generator functions}
  
  \begin{itemize}
    \item A \alert{generator function} is a function that contains the keyword \alert{yield} at
          least once in his body
    \item When you call a generator function the code is not executed - instead
          a generator object is created and returned (even if you don't have a return statement)
    \item Each call to \emph{next()} on the returned generator will make the function code 
          run until it finds a \emph{yield} statement
    \item Then the execution is paused and the value of the expression on the right 
          of \emph{yield} is returned (yielded) to the caller
    \item A further call of next will resume the execution from where it was suspended
          until the next \emph{yield} and so on
    \item Eventually, when the function body ends, \emph{StopIteration} is raised
    \item Usually generators functions contain a loop - but it's not mandatory!
  \end{itemize}
  
\end{frame}


\begin{frame}
  \frametitle{Generator functions}
  \begin{Verbatim}[label=\makebox{\url{https://github.com/lucabaldini/cmepda/tree/master/slides/latex/snippets/generator\_functions.py}},commandchars=\\\{\}]
\PY{k}{def} \PY{n+nf}{generator\PYZus{}function\PYZus{}simple}\PY{p}{(}\PY{p}{)}\PY{p}{:}
    \PY{k}{print}\PY{p}{(}\PY{l+s+s1}{\PYZsq{}}\PY{l+s+s1}{First call}\PY{l+s+s1}{\PYZsq{}}\PY{p}{)}
    \PY{k}{yield} \PY{l+m+mi}{1}
    \PY{k}{print}\PY{p}{(}\PY{l+s+s1}{\PYZsq{}}\PY{l+s+s1}{Second call}\PY{l+s+s1}{\PYZsq{}}\PY{p}{)}
    \PY{k}{yield} \PY{l+m+mi}{2}
    \PY{k}{print}\PY{p}{(}\PY{l+s+s1}{\PYZsq{}}\PY{l+s+s1}{I am about to rise a StopIteration exception...}\PY{l+s+s1}{\PYZsq{}}\PY{p}{)}

\PY{n}{gen} \PY{o}{=} \PY{n}{generator\PYZus{}function\PYZus{}simple}\PY{p}{(}\PY{p}{)} \PY{c+c1}{\PYZsh{} A generator function returns a generator}
\PY{k}{print}\PY{p}{(}\PY{n+nb}{next}\PY{p}{(}\PY{n}{gen}\PY{p}{)}\PY{p}{)} \PY{c+c1}{\PYZsh{} We stop at the first yield and get the value}
\PY{k}{print}\PY{p}{(}\PY{n+nb}{next}\PY{p}{(}\PY{n}{gen}\PY{p}{)}\PY{p}{)} \PY{c+c1}{\PYZsh{} Second yield}
\PY{n+nb}{next}\PY{p}{(}\PY{n}{gen}\PY{p}{)} \PY{c+c1}{\PYZsh{} The third next() will throw StopIteration}

[Output]
First call
1
Second call
2
I am about to rise a StopIteration exception...
Traceback (most recent call last):
  File "snippets/generator_functions.py", line 11, in <module>
    next(gen) # The third next() will throw StopIteration
StopIteration
\end{Verbatim}
\end{frame}


\begin{frame}
  \frametitle{Infinite sequence generators}
  \begin{Verbatim}[label=\makebox{\url{https://github.com/lucabaldini/cmepda/tree/master/slides/latex/snippets/fibonacci.py}},commandchars=\\\{\}]
\PY{c+c1}{\PYZsh{} Generator function that provides infinte fibonacci numbers}
\PY{k}{def} \PY{n+nf}{fibonacci}\PY{p}{(}\PY{p}{)}\PY{p}{:}
    \PY{n}{a}\PY{p}{,} \PY{n}{b} \PY{o}{=} \PY{l+m+mi}{0}\PY{p}{,} \PY{l+m+mi}{1}
    \PY{k}{while} \PY{n+nb+bp}{True}\PY{p}{:}
        \PY{k}{yield} \PY{n}{a}
        \PY{n}{a}\PY{p}{,} \PY{n}{b} \PY{o}{=} \PY{n}{b}\PY{p}{,} \PY{n}{a} \PY{o}{+} \PY{n}{b}

\PY{c+c1}{\PYZsh{} We need to impose a stop condition externally to use it}
\PY{n}{max\PYZus{}n} \PY{o}{=} \PY{l+m+mi}{7}
\PY{n}{fib\PYZus{}numbers} \PY{o}{=} \PY{p}{[}\PY{p}{]}
\PY{k}{for} \PY{n}{i}\PY{p}{,} \PY{n}{fib} \PY{o+ow}{in} \PY{n+nb}{enumerate}\PY{p}{(}\PY{n}{fibonacci}\PY{p}{(}\PY{p}{)}\PY{p}{)}\PY{p}{:}
    \PY{k}{if} \PY{n}{i} \PY{o}{\PYZgt{}}\PY{o}{=} \PY{n}{max\PYZus{}n}\PY{p}{:}
        \PY{k}{break}
    \PY{k}{else}\PY{p}{:}
        \PY{n}{fib\PYZus{}numbers}\PY{o}{.}\PY{n}{append}\PY{p}{(}\PY{n}{fib}\PY{p}{)}
\PY{k}{print}\PY{p}{(}\PY{n}{fib\PYZus{}numbers}\PY{p}{)}
      
\PY{c+c1}{\PYZsh{} Another way of doing that is using \PYZsq{}islice\PYZsq{} from itertools}
\PY{k+kn}{import} \PY{n+nn}{itertools}
\PY{c+c1}{\PYZsh{} Generator expression}
\PY{n}{fib\PYZus{}gen} \PY{o}{=} \PY{p}{(}\PY{n}{fib} \PY{k}{for} \PY{n}{fib} \PY{o+ow}{in} \PY{n}{itertools}\PY{o}{.}\PY{n}{islice}\PY{p}{(}\PY{n}{fibonacci}\PY{p}{(}\PY{p}{)}\PY{p}{,} \PY{n}{max\PYZus{}n}\PY{p}{)}\PY{p}{)}
\PY{k}{print}\PY{p}{(}\PY{n+nb}{list}\PY{p}{(}\PY{n}{fib\PYZus{}gen}\PY{p}{)}\PY{p}{)}

[Output]
[0, 1, 1, 2, 3, 5, 8]
[0, 1, 1, 2, 3, 5, 8]
\end{Verbatim}
\end{frame}


\begin{frame}
  \frametitle{Python generator functions}
  
  \begin{itemize}
    \item Python provides a number of built-in functions that return a generator from an iterable, such as:
    \begin{itemize}
      \item \emph{enumerate}: Automatic counting of iterations
      \item \emph{map}: Apply a function to the elements
      \item \emph{filter}: Return only the elements passing a given condition
      \item \emph{zip}: Return pairs of elements (requires two sequences)
      \item \emph{reversed}: Loop in the reversed order
    \end{itemize}

    \item Countless others can be found in the \alert{\emph{itertools}} library
    \begin{itemize}
      \item \emph{islice}: Slice the loop with start, stop and step
      \item \emph{takewhile}: Stop looping when a condition becomes false
      \item \emph{accumulate}: Get the results of applying the function iteratively to pair of elements
      \item \emph{chain}: Loop through many sequences one after another
      \item \emph{cycle}: Loop over the sequence repeatedly, indefinitely
      \item \emph{permutations}: Get all the permutations of a given length
      \item \emph{product}: Compute the cartesian product of iterables
      \item \emph{groupby}: Group by value of some key (function)     
      \item And so on\dots
    \end{itemize}
    
    \medskip
    
    \item Take a look at the documentation of each function to see how to 
          properly call it!
  \end{itemize}  
\end{frame}


\begin{frame}
  \frametitle{Itertools showcase}
  \begin{Verbatim}[label=\makebox{\url{https://github.com/lucabaldini/cmepda/tree/master/slides/latex/snippets/itertools\_showcase.py}},commandchars=\\\{\}]
\PY{k+kn}{from} \PY{n+nn}{itertools} \PY{k+kn}{import} \PY{n}{accumulate}\PY{p}{,} \PY{n}{product}\PY{p}{,} \PY{n}{chain}\PY{p}{,} \PY{n}{groupby}\PY{p}{,} \PY{n}{permutations}\PY{p}{,} \PY{n}{combinations}
\PY{k+kn}{import} \PY{n+nn}{operator}

\PY{n}{l1} \PY{o}{=} \PY{p}{[}\PY{l+m+mi}{1}\PY{p}{,} \PY{l+m+mi}{2}\PY{p}{,} \PY{l+m+mi}{3}\PY{p}{,} \PY{l+m+mi}{4}\PY{p}{]}
\PY{k}{print}\PY{p}{(}\PY{n+nb}{list}\PY{p}{(}\PY{n}{accumulate}\PY{p}{(}\PY{n}{l1}\PY{p}{)}\PY{p}{)}\PY{p}{)}
\PY{k}{print}\PY{p}{(}\PY{n+nb}{list}\PY{p}{(}\PY{n}{accumulate}\PY{p}{(}\PY{n}{l1}\PY{p}{,} \PY{n}{func}\PY{o}{=}\PY{n}{operator}\PY{o}{.}\PY{n}{mul}\PY{p}{)}\PY{p}{)}\PY{p}{)}
\PY{k}{print}\PY{p}{(}\PY{n+nb}{list}\PY{p}{(}\PY{n}{combinations}\PY{p}{(}\PY{n}{l1}\PY{p}{,} \PY{l+m+mi}{3}\PY{p}{)}\PY{p}{)}\PY{p}{)}

\PY{n}{l2} \PY{o}{=} \PY{p}{[}\PY{l+m+mi}{5}\PY{p}{,} \PY{l+m+mi}{6}\PY{p}{]}
\PY{k}{print}\PY{p}{(}\PY{n+nb}{list}\PY{p}{(}\PY{n}{permutations}\PY{p}{(}\PY{n}{l2}\PY{p}{,} \PY{l+m+mi}{2}\PY{p}{)}\PY{p}{)}\PY{p}{)}
\PY{k}{print}\PY{p}{(}\PY{n+nb}{list}\PY{p}{(}\PY{n}{product}\PY{p}{(}\PY{n}{l1}\PY{p}{,} \PY{n}{l2}\PY{p}{)}\PY{p}{)}\PY{p}{)}

\PY{k}{def} \PY{n+nf}{is\PYZus{}even}\PY{p}{(}\PY{n}{n}\PY{p}{)}\PY{p}{:}
    \PY{k}{return} \PY{n}{n} \PY{o}{\PYZpc{}} \PY{l+m+mi}{2} \PY{o}{==} \PY{l+m+mi}{0}

\PY{n}{l3} \PY{o}{=} \PY{n+nb}{list}\PY{p}{(}\PY{n}{chain}\PY{p}{(}\PY{n}{l1}\PY{p}{,} \PY{n}{l2}\PY{p}{)}\PY{p}{)}
\PY{c+c1}{\PYZsh{} groupby expect the list to be sorted by the grouping function}
\PY{n}{l3}\PY{o}{.}\PY{n}{sort}\PY{p}{(}\PY{n}{key}\PY{o}{=}\PY{n}{is\PYZus{}even}\PY{p}{)}
\PY{k}{for} \PY{n}{k}\PY{p}{,} \PY{n}{g} \PY{o+ow}{in} \PY{n}{groupby}\PY{p}{(}\PY{n}{l3}\PY{p}{,} \PY{n}{key}\PY{o}{=}\PY{n}{is\PYZus{}even}\PY{p}{)}\PY{p}{:}
    \PY{k}{print}\PY{p}{(}\PY{n}{k}\PY{p}{,} \PY{n+nb}{list}\PY{p}{(}\PY{n}{g}\PY{p}{)}\PY{p}{)}

[Output]
[1, 3, 6, 10]
[1, 2, 6, 24]
[(1, 2, 3), (1, 2, 4), (1, 3, 4), (2, 3, 4)]
[(5, 6), (6, 5)]
[(1, 5), (1, 6), (2, 5), (2, 6), (3, 5), (3, 6), (4, 5), (4, 6)]
False [1, 3, 5]
True [2, 4, 6]
\end{Verbatim}
\end{frame}


\begin{frame}
  \frametitle{}
  \centering \Large Lambda functions
\end{frame}


\begin{frame}
  \frametitle{Anonymous (lambda) functions}
  \begin{itemize}
    \item \alert{Anonymous functions}, or \alert{lambda functions} are a construct typical of \alert{functional programming}
    \item \url{https://en.wikipedia.org/wiki/Lambda_calculus}
    \item \url{https://en.wikipedia.org/wiki/Functional_programming}
    \item In Python a lambda function is essentially a special sintax for creating a function
          on the fly, without giving it a name
    \item They are limited to \alert{a single expression}, which is returned to the user
    \item Many of the typical uses for lambdas are already covered in python by generator expressions and comprehension,
          so this is more like a niche feature of the language
  \end{itemize}
  
\end{frame}


\begin{frame}
  \frametitle{Lambda functions}
  \begin{Verbatim}[label=\makebox{\url{https://github.com/lucabaldini/cmepda/tree/master/slides/latex/snippets/lambda.py}},commandchars=\\\{\}]
\PY{c+c1}{\PYZsh{} Here we create a lambda function and assign a name to it (ironically)}
\PY{n}{multiply} \PY{o}{=} \PY{k}{lambda} \PY{n}{x}\PY{p}{,} \PY{n}{y}\PY{p}{:} \PY{n}{x} \PY{o}{*} \PY{n}{y}
\PY{c+c1}{\PYZsh{} Use it}
\PY{k}{print}\PY{p}{(}\PY{n}{multiply}\PY{p}{(}\PY{l+m+mi}{5}\PY{p}{,} \PY{o}{\PYZhy{}}\PY{l+m+mi}{1}\PY{p}{)}\PY{p}{)}

\PY{c+c1}{\PYZsh{} Typical use is inside generator expressions}
\PY{n}{numbers} \PY{o}{=} \PY{n+nb}{range}\PY{p}{(}\PY{l+m+mi}{10}\PY{p}{)}
\PY{n}{squares} \PY{o}{=} \PY{n+nb}{list}\PY{p}{(}\PY{n+nb}{map}\PY{p}{(}\PY{k}{lambda} \PY{n}{n}\PY{p}{:} \PY{n}{n}\PY{o}{*}\PY{o}{*}\PY{l+m+mi}{2}\PY{p}{,} \PY{n}{numbers}\PY{p}{)}\PY{p}{)}
\PY{k}{print}\PY{p}{(}\PY{n}{squares}\PY{p}{)}

\PY{c+c1}{\PYZsh{} However, remeber that you can do the same with list comprehension}
\PY{n}{squares} \PY{o}{=} \PY{p}{[}\PY{n}{n}\PY{o}{*}\PY{o}{*}\PY{l+m+mi}{2} \PY{k}{for} \PY{n}{n} \PY{o+ow}{in} \PY{n}{numbers}\PY{p}{]}
\PY{k}{print}\PY{p}{(}\PY{n}{squares}\PY{p}{)}

[Output]
-5
[0, 1, 4, 9, 16, 25, 36, 49, 64, 81]
[0, 1, 4, 9, 16, 25, 36, 49, 64, 81]
\end{Verbatim}
\end{frame}


\begin{frame}
  \frametitle{Recap example: file iterator}
  \begin{Verbatim}[label=\makebox{\url{https://github.com/lucabaldini/cmepda/tree/master/slides/latex/snippets/file\_iterator.py}},commandchars=\\\{\}]
\PY{k+kn}{from} \PY{n+nn}{itertools} \PY{k+kn}{import} \PY{n}{dropwhile}

\PY{k}{class} \PY{n+nc}{LabFileIterator}\PY{p}{:}
    \PY{k}{def} \PY{n+nf+fm}{\PYZus{}\PYZus{}init\PYZus{}\PYZus{}}\PY{p}{(}\PY{n+nb+bp}{self}\PY{p}{,} \PY{n}{file\PYZus{}obj}\PY{p}{)}\PY{p}{:}
        \PY{n+nb+bp}{self}\PY{o}{.}\PY{n}{\PYZus{}lines} \PY{o}{=} \PY{n}{dropwhile}\PY{p}{(}\PY{k}{lambda} \PY{n}{line}\PY{p}{:} \PY{n}{line}\PY{o}{.}\PY{n}{startswith}\PY{p}{(}\PY{l+s+s1}{\PYZsq{}}\PY{l+s+s1}{\PYZsh{}}\PY{l+s+s1}{\PYZsq{}}\PY{p}{)}\PY{p}{,} \PY{n}{file\PYZus{}obj}\PY{p}{)}
        
    \PY{k}{def} \PY{n+nf}{\PYZus{}\PYZus{}next\PYZus{}\PYZus{}}\PY{p}{(}\PY{n+nb+bp}{self}\PY{p}{)}\PY{p}{:}
        \PY{n}{line} \PY{o}{=} \PY{n+nb}{next}\PY{p}{(}\PY{n+nb+bp}{self}\PY{o}{.}\PY{n}{\PYZus{}lines}\PY{p}{)}
        \PY{n}{values} \PY{o}{=} \PY{n}{line}\PY{o}{.}\PY{n}{strip}\PY{p}{(}\PY{l+s+s1}{\PYZsq{}}\PY{l+s+se}{\PYZbs{}n}\PY{l+s+s1}{\PYZsq{}}\PY{p}{)}\PY{o}{.}\PY{n}{split}\PY{p}{(}\PY{l+s+s1}{\PYZsq{}}\PY{l+s+s1}{ }\PY{l+s+s1}{\PYZsq{}}\PY{p}{)}
        \PY{n}{time} \PY{o}{=} \PY{n+nb}{float}\PY{p}{(}\PY{n}{values}\PY{p}{[}\PY{l+m+mi}{0}\PY{p}{]}\PY{p}{)}
        \PY{n}{tension} \PY{o}{=} \PY{n+nb}{float}\PY{p}{(}\PY{n}{values}\PY{p}{[}\PY{l+m+mi}{1}\PY{p}{]}\PY{p}{)}
        \PY{k}{return} \PY{n}{time}\PY{p}{,} \PY{n}{tension}
    
    \PY{k}{def} \PY{n+nf+fm}{\PYZus{}\PYZus{}iter\PYZus{}\PYZus{}}\PY{p}{(}\PY{n+nb+bp}{self}\PY{p}{)}\PY{p}{:}
        \PY{k}{return} \PY{n+nb+bp}{self}

\PY{k}{with} \PY{n+nb}{open}\PY{p}{(}\PY{l+s+s1}{\PYZsq{}}\PY{l+s+s1}{snippets/data/fake\PYZus{}measurements.txt}\PY{l+s+s1}{\PYZsq{}}\PY{p}{)} \PY{k}{as} \PY{n}{lab\PYZus{}data\PYZus{}file}\PY{p}{:}
    \PY{k}{try}\PY{p}{:}
        \PY{k}{for} \PY{n}{line\PYZus{}number}\PY{p}{,} \PY{p}{(}\PY{n}{time}\PY{p}{,} \PY{n}{tension}\PY{p}{)} \PY{o+ow}{in} \PY{n+nb}{enumerate}\PY{p}{(}\PY{n}{LabFileIterator}\PY{p}{(}\PY{n}{lab\PYZus{}data\PYZus{}file}\PY{p}{)}\PY{p}{)}\PY{p}{:}
            \PY{k}{print}\PY{p}{(}\PY{n}{line\PYZus{}number}\PY{p}{,} \PY{n}{time}\PY{p}{,} \PY{n}{tension}\PY{p}{)}
    \PY{k}{except} \PY{n+ne}{ValueError} \PY{k}{as} \PY{n}{e}\PY{p}{:}
        \PY{c+c1}{\PYZsh{} Here we get the wrong line number! Why?}
        \PY{k}{print}\PY{p}{(}\PY{l+s+s1}{\PYZsq{}}\PY{l+s+s1}{Line \PYZob{}\PYZcb{} error: \PYZob{}\PYZcb{}}\PY{l+s+s1}{\PYZsq{}}\PY{o}{.}\PY{n}{format}\PY{p}{(}\PY{n}{line\PYZus{}number}\PY{p}{,} \PY{n}{e}\PY{p}{)}\PY{p}{)}

[Output]
0 0.1 15.2
1 0.2 12.4
Line 1 error: could not convert string to float: 'pippo'
\end{Verbatim}
\end{frame}


\begin{frame}
  \frametitle{File iterator redone}
  \input{pygments/file_iterator_2}
\end{frame}


\begin{frame}
  \frametitle{File iterator, final version}
  \begin{Verbatim}[label=\makebox{\url{https://github.com/lucabaldini/cmepda/tree/master/slides/latex/snippets/file\_iterator\_3.py}},commandchars=\\\{\}]
\PY{k}{class} \PY{n+nc}{LabFile}\PY{p}{:}
    \PY{k}{def} \PY{n+nf+fm}{\PYZus{}\PYZus{}init\PYZus{}\PYZus{}}\PY{p}{(}\PY{n+nb+bp}{self}\PY{p}{,} \PY{n}{file\PYZus{}obj}\PY{p}{)}\PY{p}{:}
        \PY{n+nb+bp}{self}\PY{o}{.}\PY{n}{\PYZus{}file} \PY{o}{=} \PY{n}{file\PYZus{}obj}
       
    \PY{k}{def} \PY{n+nf+fm}{\PYZus{}\PYZus{}iter\PYZus{}\PYZus{}}\PY{p}{(}\PY{n+nb+bp}{self}\PY{p}{)}\PY{p}{:}
        \PY{c+c1}{\PYZsh{} This is more readible}
        \PY{k}{for} \PY{n}{i}\PY{p}{,} \PY{n}{line} \PY{o+ow}{in} \PY{n+nb}{enumerate}\PY{p}{(}\PY{n+nb+bp}{self}\PY{o}{.}\PY{n}{\PYZus{}file}\PY{p}{)}\PY{p}{:}
            \PY{k}{if} \PY{n}{line}\PY{o}{.}\PY{n}{startswith}\PY{p}{(}\PY{l+s+s1}{\PYZsq{}}\PY{l+s+s1}{\PYZsh{}}\PY{l+s+s1}{\PYZsq{}}\PY{p}{)}\PY{p}{:}
                \PY{k}{continue}
            \PY{n}{values} \PY{o}{=} \PY{n}{line}\PY{o}{.}\PY{n}{strip}\PY{p}{(}\PY{l+s+s1}{\PYZsq{}}\PY{l+s+se}{\PYZbs{}n}\PY{l+s+s1}{\PYZsq{}}\PY{p}{)}\PY{o}{.}\PY{n}{split}\PY{p}{(}\PY{l+s+s1}{\PYZsq{}}\PY{l+s+s1}{ }\PY{l+s+s1}{\PYZsq{}}\PY{p}{)}
            \PY{k}{try}\PY{p}{:}
                \PY{n}{time} \PY{o}{=} \PY{n+nb}{float}\PY{p}{(}\PY{n}{values}\PY{p}{[}\PY{l+m+mi}{0}\PY{p}{]}\PY{p}{)}
                \PY{n}{tension} \PY{o}{=} \PY{n+nb}{float}\PY{p}{(}\PY{n}{values}\PY{p}{[}\PY{l+m+mi}{1}\PY{p}{]}\PY{p}{)}
            \PY{k}{except} \PY{n+ne}{ValueError} \PY{k}{as} \PY{n}{e}\PY{p}{:}
                \PY{k}{print}\PY{p}{(}\PY{l+s+s1}{\PYZsq{}}\PY{l+s+s1}{Line \PYZob{}\PYZcb{} error: \PYZob{}\PYZcb{}}\PY{l+s+s1}{\PYZsq{}}\PY{o}{.}\PY{n}{format}\PY{p}{(}\PY{n}{i}\PY{p}{,} \PY{n}{e}\PY{p}{)}\PY{p}{)}
                \PY{k}{continue}
            \PY{k}{yield} \PY{n}{time}\PY{p}{,} \PY{n}{tension}

\PY{k}{with} \PY{n+nb}{open}\PY{p}{(}\PY{l+s+s1}{\PYZsq{}}\PY{l+s+s1}{snippets/data/fake\PYZus{}measurements.txt}\PY{l+s+s1}{\PYZsq{}}\PY{p}{)} \PY{k}{as} \PY{n}{lab\PYZus{}data\PYZus{}file}\PY{p}{:}
    \PY{k}{for} \PY{n}{time}\PY{p}{,} \PY{n}{tension} \PY{o+ow}{in} \PY{n}{LabFile}\PY{p}{(}\PY{n}{lab\PYZus{}data\PYZus{}file}\PY{p}{)}\PY{p}{:}
        \PY{k}{print}\PY{p}{(}\PY{n}{time}\PY{p}{,} \PY{n}{tension}\PY{p}{)}

[Output]
0.1 15.2
0.2 12.4
Line 3 error: could not convert string to float: 'pippo'
0.4 13.2
\end{Verbatim}
\end{frame}

\end{document}

