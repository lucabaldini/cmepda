\documentclass[9pt]{beamer}
\usetheme{cmepda}

\usepackage[utf8]{inputenc}
\usepackage[T1]{fontenc}


\title{numpy and scipy (1/2)}
\subtitle{Computing Methods for Experimental Physics and Data Analysis}
\date{Compiled on \today}
\author{L. Baldini}
\institute[UNIPI and INFN]{Universit\`a and INFN--Pisa}
\email{luca.baldini@pi.infn.it}


\begin{document}


\titleframe

\begin{frame}
  \frametitle{Introduction}
  \begin{itemize}
  \item Python is notorious for coming \emph{with batteries included}\ldots
  \item \ldots but as a data scientist numpy and and scipy will be your
    best friends!
  \item Among (many) other things, numpy offers:
    \begin{itemize}
    \item a powerful n-dimensional array object
    \item mathematical functions that interoperate natively with arrays
    \end{itemize}
  \item And scipy provides:
    \begin{itemize}
    \item integration
    \item optimization (a.k.a. fitting)
    \item interpolation
    \item signal processing
    \end{itemize}
  \end{itemize}
\end{frame}


\begin{frame}
  \frametitle{numpy arrays}
  \input{pygments/numpy_arrays}
\end{frame}


\begin{frame}
  \frametitle{numpy arrays vs. Python lists}
  \input{pygments/numpy_arrays_vs_lists}
  
  \begin{itemize}
  \item arrays and lists are fundamentally different objects
    \begin{itemize}
    \item different footprint in memory, operate at different speed
    \item arrays are homogeneous, lists don't need to
    \item arrays offer a much more powerful indexing/slicing
    \item arrays interoperate with numpy mathematical functions
    \end{itemize}
  \end{itemize}
\end{frame}


\begin{frame}
  \frametitle{Broadcasting}
  \input{pygments/numpy_arrays_broadcasting}
  
  \begin{itemize}
  \item Under certain conditions, numpy can make operations on arrays of
    different shape
  \item This is extremely useful when vectorizing problems 
  \end{itemize}
\end{frame}


\begin{frame}
  \frametitle{Mathematical functions in numpy}
  \input{pygments/numpy_functions}
  
  \begin{itemize}
  \item numpy mathematical functions interoperate natively with arrays
    \begin{itemize}
    \item (and work on plain old numbers, too)
    \end{itemize}
  \end{itemize}
\end{frame}


\begin{frame}
  \frametitle{Arrays and masks}
  \input{pygments/numpy_masks}
  
  \begin{itemize}
  \item Masks are a powerful tool in numpy
  \item They can replace conditional expressions in a for loop in
    vectorization context (more on this later)
  \end{itemize}
\end{frame}


\begin{frame}[fragile]
  \frametitle{Digression: pseudo-random number generators}

  \begin{Verbatim}
    import random
    x = random.random()
  \end{Verbatim}

  \medskip
  
  \begin{itemize}
  \item Every programming language comes with a Pseudo Random Number
    Generator (PRNG)
    \begin{itemize}
    \item Python is no exception:
      \url{https://docs.python.org/3/library/random.html}
    \item Mersenne-Twister, 53-bit precision, period of $2^{19937} - 1$. 
    \end{itemize}
  \item PRNGs are an interesting (and fun) subject by themselves:
    \begin{itemize}
    \item Donald E. Knuth, \emph{The Art of Computer Programming, Volume 2: Seminumerical Algorithms}, 3rd Edition 
    \item M. Matsumoto and T. Nishimura, \emph{Mersenne Twister: A 623-dimensionally equidistributed uniform pseudorandom number generator}, ACM Transactions on Modeling and Computer Simulation Vol. 8, No. 1, January pp.3--30 1998.
    \end{itemize}
  \item A PRNG produces random floats uniformly in $[0.0,~1.0)$.
  \end{itemize}
\end{frame}


\begin{frame}
  \frametitle{Vectorization}
  \framesubtitle{a.k.a. avoid explicit for loops in Python whenever you can}
  \input{pygments/vectorization}
\end{frame}


\begin{frame}
  \frametitle{How does vectorizaion work?}
  \begin{itemize}
  \item Python is know to be slooow
    \begin{itemize}
    \item This is the price you pay for being so beautiful and flexible
    \end{itemize}
  \item Does it matter? If depends\ldots
    \begin{itemize}
    \item If you are parsing a text file or fetching a web page probably not
    \item If you are performing a CPU-intensive processing on a TB of data
      probably yes
    \end{itemize}
  \item What's so magic in using numpy?
  \item numpy is written in C as a Python extension
    \begin{itemize}
    \item Routines are highly optimized to cruch numbers
    \item When you perform an array operation in Python you are actually
      executing optimized C code
    \end{itemize}
  \item \alert{Basic message: avoid for loops in pure Python when
    crunching numbers}
  \end{itemize}
\end{frame}


\begin{frame}
  \frametitle{How do I throw PRN with arbitrary pdf?}
  \framesubtitle{Hit or miss}
  \centering\includegraphics[width=0.5\textwidth]{hitmiss}

  \begin{itemize}
  \item Hit or miss, aka acceptance/rejection method:
    \begin{itemize}
    \item Enclose your pdf in a rectangle
    \item Throw a $x$ and a $y$
    \item Accept $x$ if $y \leq f(x)$
    \end{itemize}
  \item \alert{This is horrible---please don't use it!}
  \end{itemize}
\end{frame}


\begin{frame}
  \frametitle{How do I throw PRN with arbitrary pdf?}
  \framesubtitle{Inverse transform}
  \begin{itemize}
  \item Probability density function (pdf)
    $$
    p(x) \quad (\ge 0)
    $$
  \item Cumulative function (cf)
    $$
    F(x) = \int_{-\infty}^{x} p(x') dx'
    $$
  \item Percent-point function (ppf)
    $$
    x = F^{-1}(q)
    $$
  \item \alert{Awesome fact: if $q$ is uniformly distributed in $[0, 1]$,
    then $x = F^{-1}(q)$ is distributed according to $p(x)$!}
  \end{itemize}
\end{frame}


\begin{frame}
  \frametitle{An interesting object: splines}
  \centering\includegraphics[width=0.60\textwidth]{spline}

  \begin{itemize}
  \item Defined piecewsise by polinomials of degree k ($k = 3$ fairly popular)
    \begin{itemize}
    \item Interpolating: passing through a set of pre-defined points
    \item First $k-1$ derivatives continuos at the control points
    \end{itemize}
  \item Superior to polynomial interpolation or curve fitting in many cases
  \end{itemize}
\end{frame}


\begin{frame}
  \frametitle{Splines: construction and properties}
  \input{pygments/spline}
  
  \begin{itemize}
  \item Evaluation is fairly inexpensive
    \begin{itemize}
    \item If the input $x$-array is sorted can do a binary search in
      \alert{O(log(N)) complexity}
    \end{itemize}
  \item Derivatives and integrals are easy
    \begin{itemize}
    \item Can be calculated \emph{exactly} by means of elementary
      arithmetic operations
    \end{itemize}
  \end{itemize}
\end{frame}


\begin{frame}
  \frametitle{References}
  \scriptsize
  \begin{itemize}
  \item \url{https://numpy.org/}
  \item \url{https://www.scipy.org/}
  \item \url{https://docs.scipy.org/doc/numpy/reference/arrays.indexing.html}
  \item \url{https://docs.scipy.org/doc/numpy/user/basics.broadcasting.html}
  \item \url{https://docs.scipy.org/doc/scipy/reference/interpolate.html}
  \end{itemize}
\end{frame}



\end{document}
