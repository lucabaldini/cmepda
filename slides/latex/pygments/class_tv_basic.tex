\begin{Verbatim}[label=\makebox{\url{https://github.com/lucabaldini/cmepda/tree/master/slides/latex/snippets/class\_tv\_basic.py}},commandchars=\\\{\}]
\PY{c+c1}{\PYZsh{} Here we define the class}
\PY{k}{class} \PY{n+nc}{Television}\PY{p}{:}
    \PY{l+s+sd}{\PYZdq{}\PYZdq{}\PYZdq{} Television class. I will follow the convention of starting class names}
\PY{l+s+sd}{    with an uppercase. \PYZdq{}\PYZdq{}\PYZdq{}}
    \PY{k}{pass} \PY{c+c1}{\PYZsh{} oops we have no code yet!}

\PY{l+s+sd}{\PYZdq{}\PYZdq{}\PYZdq{}To create instances of a class in python we use the parenthesis operator \PYZsq{}()\PYZsq{}.}
\PY{l+s+sd}{The syntax is similar to calling a function \PYZhy{}\PYZhy{} which is actually what is }
\PY{l+s+sd}{happening behind the scenes, as we will see later\PYZdq{}\PYZdq{}\PYZdq{}}
\PY{n}{my\PYZus{}television} \PY{o}{=} \PY{n}{Television}\PY{p}{(}\PY{p}{)} \PY{c+c1}{\PYZsh{} my\PYZus{}television is an instance of the class Television}

\PY{k}{print}\PY{p}{(}\PY{n+nb}{type}\PY{p}{(}\PY{n}{my\PYZus{}television}\PY{p}{)}\PY{p}{)} \PY{c+c1}{\PYZsh{} Check its type}

\PY{n}{your\PYZus{}television} \PY{o}{=} \PY{n}{Television}\PY{p}{(}\PY{p}{)} \PY{c+c1}{\PYZsh{} And this is another instance}

\PY{c+c1}{\PYZsh{} Let\PYZsq{}s check that they are really two different objects}
\PY{k}{print}\PY{p}{(}\PY{n}{my\PYZus{}television} \PY{o+ow}{is} \PY{o+ow}{not} \PY{n}{your\PYZus{}television}\PY{p}{)}

[Output]
<class '__main__.Television'>
True
\end{Verbatim}