\begin{Verbatim}[label=\makebox{\url{https://github.com/lucabaldini/cmepda/tree/master/slides/latex/snippets/class\_methods.py}},commandchars=\\\{\}]
\PY{k}{class} \PY{n+nc}{Television}\PY{p}{:}
    \PY{l+s+sd}{\PYZdq{}\PYZdq{}\PYZdq{} Class describing a televsion.}
\PY{l+s+sd}{    \PYZdq{}\PYZdq{}\PYZdq{}}
    \PY{k}{def} \PY{n+nf}{turn\PYZus{}on}\PY{p}{(}\PY{n+nb+bp}{self}\PY{p}{,} \PY{n}{channel}\PY{o}{=}\PY{l+m+mi}{1}\PY{p}{)}\PY{p}{:} \PY{c+c1}{\PYZsh{} Class method}
        \PY{l+s+sd}{\PYZdq{}\PYZdq{}\PYZdq{}All the class methods get the object instance as their first argument.}
\PY{l+s+sd}{        It is customary to call this argument \PYZsq{}self\PYZsq{}, though is not required}
\PY{l+s+sd}{        by the language rules (you can call it \PYZsq{}pippo\PYZsq{} and it will work}
\PY{l+s+sd}{        just as well)}
\PY{l+s+sd}{        \PYZdq{}\PYZdq{}\PYZdq{}}
        \PY{n+nb}{print}\PY{p}{(}\PY{l+s+sa}{f}\PY{l+s+s1}{\PYZsq{}}\PY{l+s+s1}{Turning on }\PY{l+s+si}{\PYZob{}}\PY{n+nb+bp}{self}\PY{l+s+si}{\PYZcb{}}\PY{l+s+s1}{\PYZsq{}}\PY{p}{)}
        \PY{n+nb}{print}\PY{p}{(}\PY{l+s+sa}{f}\PY{l+s+s1}{\PYZsq{}}\PY{l+s+s1}{Showing channel }\PY{l+s+si}{\PYZob{}}\PY{n}{channel}\PY{l+s+si}{\PYZcb{}}\PY{l+s+s1}{\PYZsq{}}\PY{p}{)}

\PY{n}{tv} \PY{o}{=} \PY{n}{Television}\PY{p}{(}\PY{p}{)}
\PY{c+c1}{\PYZsh{} Class methods and members are accessed through the \PYZsq{}.\PYZsq{} (dot) operator}
\PY{c+c1}{\PYZsh{} You must not pass the \PYZsq{}self\PYZsq{} argument, it is added automatically!}
\PY{n}{tv}\PY{o}{.}\PY{n}{turn\PYZus{}on}\PY{p}{(}\PY{n}{channel}\PY{o}{=}\PY{l+m+mi}{3}\PY{p}{)}

[Output]
Turning on <__main__.Television object at 0x7f988399d8d0>
Showing channel 3
\end{Verbatim}