\documentclass[9pt]{beamer}
\usetheme{cmepda}

\usepackage[utf8]{inputenc}
\usepackage[T1]{fontenc}

\graphicspath{{figures/}} 


\title{Advanced python features - part II}
\subtitle{Computing Methods for Experimental Physics and Data Analysis}
\date{Compiled on \today}
\author{A. Manfreda}
\institute[INFN]{INFN--Pisa}
\email{alberto.manfreda@pi.infn.it}


\begin{document}


\titleframe


\begin{frame}
  \frametitle{Functions inside functions}
  \begin{itemize}
    \item Functions in python are \alert{first class object}
    \item The name is a bit misleading, but what it actually means is that functions
          can be passed as argument to other functions and returned as result from
          other functions
    \item This shouldn't surprise you much: functions are objects of a
          '\emph{function}' class, so they behave like any other vairable in
          Python
    \item Another thing you can (and sometimes want to) do is \emph{defining} a
          function inside another.
    \item Let's see how it works
  \end{itemize}
  
\end{frame}


\begin{frame}
  \frametitle{Functions inside functions}
  \input{pygments/inner_outer}
\end{frame}


\begin{frame}
  \frametitle{Colsures and free variables}
  \begin{itemize}
    \item When a function is created inside another function it has access to
          the local variables of the outer function, even after its scope ended
    \smallskip
    \item This is techincally possible because those varibales are kept in a special
          space of memory, the \alert{closure} of the inner function
    \smallskip
    \item Such variables are called \alert{free variables}
    \smallskip
    \item In this way, a function can mantain a \alert{state} within its closure
    \smallskip
    \item This makes possible the use of functions for tasks that in other
          languages are reserved to classes
    \smallskip
    \item Let's take a look at the following example
  \end{itemize}
\end{frame}


\begin{frame}
  \frametitle{An inefficient rotation}
  \begin{Verbatim}[label=\makebox{\url{https://github.com/lucabaldini/cmepda/tree/master/slides/latex/snippets/rotation\_naive.py}},commandchars=\\\{\}]
\PY{k+kn}{import} \PY{n+nn}{numpy}

\PY{k}{def} \PY{n+nf}{rotate}\PY{p}{(}\PY{n}{x}\PY{p}{,} \PY{n}{y}\PY{p}{,} \PY{n}{theta}\PY{p}{)}\PY{p}{:}
    \PY{l+s+sd}{\PYZdq{}\PYZdq{}\PYZdq{} Naive implementation. This works, but is inefficient \PYZhy{} we have to}
\PY{l+s+sd}{    calculate cos(theta) and sin(theta) for each pair (x,y)! \PYZdq{}\PYZdq{}\PYZdq{}}
    \PY{n}{c} \PY{o}{=} \PY{n}{numpy}\PY{o}{.}\PY{n}{cos}\PY{p}{(}\PY{n}{theta}\PY{p}{)}
    \PY{n}{s} \PY{o}{=} \PY{n}{numpy}\PY{o}{.}\PY{n}{sin}\PY{p}{(}\PY{n}{theta}\PY{p}{)}
    \PY{n}{x\PYZus{}rot} \PY{o}{=} \PY{n}{c} \PY{o}{*} \PY{n}{x} \PY{o}{\PYZhy{}} \PY{n}{s} \PY{o}{*} \PY{n}{y}
    \PY{n}{y\PYZus{}rot} \PY{o}{=} \PY{n}{s} \PY{o}{*} \PY{n}{x} \PY{o}{+} \PY{n}{c} \PY{o}{*} \PY{n}{y}
    \PY{k}{return} \PY{n}{x\PYZus{}rot}\PY{p}{,} \PY{n}{y\PYZus{}rot}

\PY{n}{x}\PY{p}{,} \PY{n}{y} \PY{o}{=} \PY{l+m+mf}{1.}\PY{p}{,} \PY{l+m+mf}{0.}
\PY{n}{theta} \PY{o}{=} \PY{n}{numpy}\PY{o}{.}\PY{n}{pi}\PY{o}{/}\PY{l+m+mi}{4}
\PY{k}{print}\PY{p}{(}\PY{n}{rotate}\PY{p}{(}\PY{n}{x}\PY{p}{,} \PY{n}{y}\PY{p}{,} \PY{n}{theta}\PY{p}{)}\PY{p}{)} \PY{c+c1}{\PYZsh{} Rotation of pi/2}
    
\PY{k}{def} \PY{n+nf}{efficient\PYZus{}rotate}\PY{p}{(}\PY{n}{x}\PY{p}{,} \PY{n}{y}\PY{p}{,} \PY{n}{c\PYZus{}theta}\PY{p}{,} \PY{n}{s\PYZus{}theta}\PY{p}{)}\PY{p}{:}
    \PY{l+s+sd}{\PYZdq{}\PYZdq{}\PYZdq{} Efficient rotation. The user gives cos and sin, so they are not}
\PY{l+s+sd}{    calculated for each pixel \PYZdq{}\PYZdq{}\PYZdq{}}
    \PY{n}{x\PYZus{}rot} \PY{o}{=} \PY{n}{c} \PY{o}{*} \PY{n}{x} \PY{o}{\PYZhy{}} \PY{n}{s} \PY{o}{*} \PY{n}{y}
    \PY{n}{y\PYZus{}rot} \PY{o}{=} \PY{n}{s} \PY{o}{*} \PY{n}{x} \PY{o}{+} \PY{n}{c} \PY{o}{*} \PY{n}{y}
    \PY{k}{return} \PY{n}{x\PYZus{}rot}\PY{p}{,} \PY{n}{y\PYZus{}rot}
    
\PY{n}{c} \PY{o}{=} \PY{n}{numpy}\PY{o}{.}\PY{n}{cos}\PY{p}{(}\PY{n}{theta}\PY{p}{)}
\PY{n}{s} \PY{o}{=} \PY{n}{numpy}\PY{o}{.}\PY{n}{sin}\PY{p}{(}\PY{n}{theta}\PY{p}{)}
\PY{k}{print}\PY{p}{(}\PY{n}{efficient\PYZus{}rotate}\PY{p}{(}\PY{n}{x}\PY{p}{,} \PY{n}{y}\PY{p}{,} \PY{n}{c}\PY{p}{,} \PY{n}{s}\PY{p}{)}\PY{p}{)} \PY{c+c1}{\PYZsh{} The syntax is very ugly!}

[Output]
(0.7071067811865476, 0.7071067811865475)
(0.7071067811865476, 0.7071067811865475)
\end{Verbatim}
\end{frame}


\begin{frame}
  \frametitle{An efficient rotation}
  \begin{Verbatim}[label=\makebox{\url{https://github.com/lucabaldini/cmepda/tree/master/slides/latex/snippets/rotator.py}},commandchars=\\\{\}]
\PY{k+kn}{import} \PY{n+nn}{numpy}

\PY{k}{def} \PY{n+nf}{create\PYZus{}rotator}\PY{p}{(}\PY{n}{theta}\PY{p}{)}\PY{p}{:}
    \PY{l+s+sd}{\PYZdq{}\PYZdq{}\PYZdq{} Efficient rotation. Cosinus and sinus values are saved in the closure,}
\PY{l+s+sd}{    so that they are computed exactly once.\PYZdq{}\PYZdq{}\PYZdq{}}
    \PY{n}{c} \PY{o}{=} \PY{n}{numpy}\PY{o}{.}\PY{n}{cos}\PY{p}{(}\PY{n}{theta}\PY{p}{)}
    \PY{n}{s} \PY{o}{=} \PY{n}{numpy}\PY{o}{.}\PY{n}{sin}\PY{p}{(}\PY{n}{theta}\PY{p}{)}
    \PY{k}{def} \PY{n+nf}{rotate}\PY{p}{(}\PY{n}{x}\PY{p}{,} \PY{n}{y}\PY{p}{)}\PY{p}{:}
        \PY{n}{x\PYZus{}rot} \PY{o}{=} \PY{n}{c} \PY{o}{*} \PY{n}{x} \PY{o}{\PYZhy{}} \PY{n}{s} \PY{o}{*} \PY{n}{y}
        \PY{n}{y\PYZus{}rot} \PY{o}{=} \PY{n}{s} \PY{o}{*} \PY{n}{x} \PY{o}{+} \PY{n}{c} \PY{o}{*} \PY{n}{y}
        \PY{k}{return} \PY{n}{x\PYZus{}rot}\PY{p}{,} \PY{n}{y\PYZus{}rot}
    \PY{k}{return} \PY{n}{rotate}

\PY{n}{x}\PY{p}{,} \PY{n}{y} \PY{o}{=} \PY{l+m+mf}{1.}\PY{p}{,} \PY{l+m+mf}{0.}
\PY{n}{theta} \PY{o}{=} \PY{n}{numpy}\PY{o}{.}\PY{n}{pi}\PY{o}{/}\PY{l+m+mi}{4}
\PY{n}{rotate\PYZus{}by\PYZus{}theta} \PY{o}{=} \PY{n}{create\PYZus{}rotator}\PY{p}{(}\PY{n}{theta}\PY{p}{)}
\PY{k}{print}\PY{p}{(}\PY{n}{rotate\PYZus{}by\PYZus{}theta}\PY{p}{(}\PY{n}{x}\PY{p}{,} \PY{n}{y}\PY{p}{)}\PY{p}{)}

[Output]
(0.7071067811865476, 0.7071067811865475)
\end{Verbatim}
\end{frame}



\begin{frame}
  \frametitle{A caveat about free variables}
  \begin{itemize}  
    \item Note: if you \emph{assign} to a free variable in the inner function, 
          by default a new, local variable is created instead!
    \medskip
    \item To avoid this you have to explictly declare that you want to access the variable in
          the closure using the \alert{nonlocal} keyword
    \medskip
    \item Remember: \emph{'Explicit is better then implicit'}
  \end{itemize}
\end{frame}



\begin{frame}
  \frametitle{Free variables - a mistake to avoid}
  \input{pygments/closure_wrong}
\end{frame}


\begin{frame}
  \frametitle{Free variables - the correct way}
  \input{pygments/closure_right}
\end{frame}


\begin{frame}
  \frametitle{Wrapping functions}
  \begin{itemize}
    \item The typical use of defining a function inside a function is to create
          a \alert{wrapper}
    \item A wrapper is a function that calls another one adding a layer
          of functionalities in between - for example it may do some pre-process
          of the input, or change the output in some way, or measure the 
          execution time or whatever we want
    \item The techinque for creating a wrapper fucntion in Python is:
    \begin{itemize}
       \item Pass the function that we want to wrap as argument of the outer
             function
       \item Inside the outer function we define an inner function, which is the
             actual wrapper
       \item The wrapper calls the wrapped function and adds its functionalities,
             before and/or after the call. It may return the same output or a 
             manipulated one.
       \item Then from the outer fucntion we return the wrapper
    \end{itemize}
  \end{itemize}
  
\end{frame}


\begin{frame}
  \frametitle{Wrapper}
  \begin{Verbatim}[label=\makebox{\url{https://github.com/lucabaldini/cmepda/tree/master/slides/latex/snippets/wrapper.py}},commandchars=\\\{\}]
\PY{k}{def} \PY{n+nf}{some\PYZus{}function}\PY{p}{(}\PY{n}{a}\PY{p}{,} \PY{n}{b}\PY{p}{)}\PY{p}{:}
    \PY{k}{print}\PY{p}{(}\PY{l+s+s1}{\PYZsq{}}\PY{l+s+s1}{Executing \PYZob{}\PYZcb{} x \PYZob{}\PYZcb{}}\PY{l+s+s1}{\PYZsq{}}\PY{o}{.}\PY{n}{format}\PY{p}{(}\PY{n}{a}\PY{p}{,} \PY{n}{b}\PY{p}{)}\PY{p}{)}
    \PY{k}{return} \PY{n}{a} \PY{o}{*} \PY{n}{b}
    
\PY{k}{def} \PY{n+nf}{add\PYZus{}n\PYZus{}wrapper}\PY{p}{(}\PY{n}{func}\PY{p}{,} \PY{n}{n}\PY{p}{)}\PY{p}{:} \PY{c+c1}{\PYZsh{} We take the wrapped function as argument}
    \PY{l+s+sd}{\PYZdq{}\PYZdq{}\PYZdq{} This wrapper adds n to the result of the wrapped function\PYZdq{}\PYZdq{}\PYZdq{}}
    
    \PY{k}{def} \PY{n+nf}{wrapper}\PY{p}{(}\PY{o}{*}\PY{n}{args}\PY{p}{,} \PY{o}{*}\PY{o}{*}\PY{n}{kwargs}\PY{p}{)}\PY{p}{:} 
       \PY{l+s+sd}{\PYZdq{}\PYZdq{}\PYZdq{}We passs the arguments as *arg, **kwargs, because this is the most}
\PY{l+s+sd}{       general form in Python: we can collect any combination of arguments like}
\PY{l+s+sd}{       that. Note that we have access to both \PYZsq{}func\PYZsq{} and \PYZsq{}n\PYZsq{}, as they are stored}
\PY{l+s+sd}{       in the closure of \PYZsq{}wrapper\PYZsq{}\PYZdq{}\PYZdq{}\PYZdq{}}
       \PY{n}{result} \PY{o}{=} \PY{n}{func}\PY{p}{(}\PY{o}{*}\PY{n}{args}\PY{p}{,} \PY{o}{*}\PY{o}{*}\PY{n}{kwargs}\PY{p}{)} \PY{c+c1}{\PYZsh{} Pass the arguments to the wrapped fucntion}
       \PY{k}{print}\PY{p}{(}\PY{l+s+s1}{\PYZsq{}}\PY{l+s+s1}{Adding \PYZob{}\PYZcb{}}\PY{l+s+s1}{\PYZsq{}}\PY{o}{.}\PY{n}{format}\PY{p}{(}\PY{n}{n}\PY{p}{)}\PY{p}{)}
       \PY{k}{return} \PY{n}{result} \PY{o}{+} \PY{n}{n} \PY{c+c1}{\PYZsh{} Return a modified result in this case}
       
    \PY{k}{return} \PY{n}{wrapper} \PY{c+c1}{\PYZsh{} From add\PYZus{}n\PYZus{}wrapper we return the wrapper}

\PY{n}{function\PYZus{}plus\PYZus{}five} \PY{o}{=} \PY{n}{add\PYZus{}n\PYZus{}wrapper}\PY{p}{(}\PY{n}{some\PYZus{}function}\PY{p}{,} \PY{l+m+mi}{5}\PY{p}{)}
\PY{k}{print}\PY{p}{(}\PY{l+s+s1}{\PYZsq{}}\PY{l+s+s1}{Result = \PYZob{}\PYZcb{}}\PY{l+s+s1}{\PYZsq{}}\PY{o}{.}\PY{n}{format}\PY{p}{(}\PY{n}{function\PYZus{}plus\PYZus{}five}\PY{p}{(}\PY{l+m+mi}{2}\PY{p}{,} \PY{l+m+mi}{3}\PY{p}{)}\PY{p}{)}\PY{p}{)}

[Output]
Executing 2 x 3
Adding 5
Result = 11
\end{Verbatim}
\end{frame}


\begin{frame}
  \frametitle{Decorators}
  \begin{itemize}
    \item Often, when you wrap a function, you don't want to change
          it's name, so you reassign the wrapped function to its old name
    \item In fact, this techinque is so common that python introduced a special
          syntax for it: decorators
    \item A decorated function has simply the name of the wrapper added with
          a '@' on top of its declaration
  \end{itemize}
  
\end{frame}


\begin{frame}
  \frametitle{Decorators}
  \input{pygments/decorator}
\end{frame}


\begin{frame}
  \frametitle{A decorator to measure execution time}
  \input{pygments/time_measuring_decor}
\end{frame}


\begin{frame}
  \frametitle{The @classmethod decorator}
  \begin{itemize}
    \item We have already seen a built-in Python decorator: \emph{@property}
    \item We used that to get proper encapsulation
    \item There is another built-in decorator one which is very useful for classes: \alert{\emph{@classmethod}}
    \item A classmethod is like a class attribute: you don't need an instance to
          use it
    \item A class method can access class attributes but not instance attributes
    \item The main use for class methods is to provide \alert{alternate constructors}
  \end{itemize}
\end{frame}


\begin{frame}
  \frametitle{Class method}
  \input{pygments/classmethod}
\end{frame}


\begin{frame}
  \frametitle{The @staticmethod decorator}
  \begin{itemize}
    \item Another built-in decorator (though less used than \emph{@classmethod})
          is \alert{\emph{@staticmethod}}
    \smallskip
    \item A static method is a method that does not receive the class or instance
          as first argument
    \smallskip
    \item Because of that, a static method does not alter the state of the class
    \smallskip
    \item In some sense a static method is only loosely coupled to the class - 
          it is defined inside the class body just for convenience (maybe for
          semantical proximity) but it could be defined outside the class as well
  \end{itemize}
\end{frame}


\begin{frame}
  \frametitle{Static method}
  \begin{Verbatim}[label=\makebox{\url{https://github.com/lucabaldini/cmepda/tree/master/slides/latex/snippets/staticmethod.py}},commandchars=\\\{\}]
\PY{k+kn}{import} \PY{n+nn}{math} 

\PY{k}{class} \PY{n+nc}{Angle}\PY{p}{:}
    
    \PY{c+c1}{\PYZsh{} A bunch of useful methods....}
    
    \PY{n+nd}{@staticmethod}
    \PY{k}{def} \PY{n+nf}{rad2deg}\PY{p}{(}\PY{n}{rad}\PY{p}{)}\PY{p}{:}
        \PY{c+c1}{\PYZsh{} No self argument here}
        \PY{k}{return} \PY{n}{rad} \PY{o}{*} \PY{l+m+mf}{180.}\PY{o}{/}\PY{n}{math}\PY{o}{.}\PY{n}{pi}
    
    \PY{n+nd}{@staticmethod}
    \PY{k}{def} \PY{n+nf}{deg2rad}\PY{p}{(}\PY{n}{deg}\PY{p}{)}\PY{p}{:}
        \PY{c+c1}{\PYZsh{} No self argument here}
        \PY{k}{return} \PY{n}{deg} \PY{o}{*} \PY{n}{math}\PY{o}{.}\PY{n}{pi} \PY{o}{/} \PY{l+m+mf}{180.}
        
\PY{k}{print}\PY{p}{(}\PY{n}{Angle}\PY{o}{.}\PY{n}{rad2deg}\PY{p}{(}\PY{n}{math}\PY{o}{.}\PY{n}{pi}\PY{o}{/}\PY{l+m+mi}{2}\PY{p}{)}\PY{p}{)}
\PY{k}{print}\PY{p}{(}\PY{n}{Angle}\PY{o}{.}\PY{n}{deg2rad}\PY{p}{(}\PY{l+m+mf}{45.}\PY{p}{)}\PY{p}{)}

[Output]
90.0
0.7853981633974483
\end{Verbatim}
\end{frame}


\begin{frame}
  \frametitle{Passing arguments to a decorator}
  \begin{itemize}
    \item Making a decorator that accepts arguments is somewhat more complex
    \smallskip
    \item The basic idea is that you need to add yet another level: a 
          \emph{decorator factory} function
    \smallskip
    \item So you now end up with three levels:
    \smallskip
    \begin{itemize}
      \item The decorator factory that takes the parameters for the decorators, creates it and retruns it
      \item The decorator that takes as input a function and returns the wrapper
      \item The wrapper that implements the actual additional functionalities;
            usually takes as input the same parameters as the decorated/wrapped
            function and returns its results (if any)
    \end{itemize}
  \end{itemize}
\end{frame}


\begin{frame}
  \frametitle{Decorception}
  \begin{Verbatim}[label=\makebox{\url{https://github.com/lucabaldini/cmepda/tree/master/slides/latex/snippets/decorator\_factory.py}},commandchars=\\\{\}]
\PY{k+kn}{from} \PY{n+nn}{functools} \PY{k+kn}{import} \PY{n}{wraps}

\PY{c+c1}{\PYZsh{} This is how a generic decorator with arguments looks like}

\PY{k}{def} \PY{n+nf}{decorator\PYZus{}factory}\PY{p}{(}\PY{o}{*}\PY{n}{params}\PY{p}{)}\PY{p}{:} \PY{c+c1}{\PYZsh{} The signature can be anything}
    \PY{k}{def} \PY{n+nf}{decorator}\PY{p}{(}\PY{n}{function}\PY{p}{)}\PY{p}{:}
        \PY{n+nd}{@wraps}\PY{p}{(}\PY{n}{function}\PY{p}{)}
        \PY{k}{def} \PY{n+nf}{wrapper}\PY{p}{(}\PY{o}{*}\PY{n}{args}\PY{p}{,} \PY{o}{*}\PY{o}{*}\PY{n}{kwargs}\PY{p}{)}\PY{p}{:}
            \PY{k}{print}\PY{p}{(}\PY{n}{f}\PY{l+s+s1}{\PYZsq{}}\PY{l+s+s1}{Decorator arguments: \PYZob{}params\PYZcb{}}\PY{l+s+s1}{\PYZsq{}}\PY{p}{)}
            \PY{c+c1}{\PYZsh{} some preprocess here}
            \PY{n}{result} \PY{o}{=} \PY{n}{function}\PY{p}{(}\PY{o}{*}\PY{n}{args}\PY{p}{,} \PY{o}{*}\PY{o}{*}\PY{n}{kwargs}\PY{p}{)}
            \PY{c+c1}{\PYZsh{} some post\PYZhy{}process here}
            \PY{k}{return} \PY{n}{result}
        \PY{k}{return} \PY{n}{wrapper}
    \PY{k}{return} \PY{n}{decorator}

\PY{c+c1}{\PYZsh{} usage}
\PY{n+nd}{@decorator\PYZus{}factory}\PY{p}{(}\PY{l+m+mi}{1}\PY{p}{,} \PY{l+m+mi}{2}\PY{p}{,} \PY{l+m+mi}{3}\PY{p}{)}
\PY{k}{def} \PY{n+nf}{f}\PY{p}{(}\PY{n}{x}\PY{p}{)}\PY{p}{:}
    \PY{k}{return} \PY{n}{x}\PY{o}{*}\PY{o}{*}\PY{l+m+mi}{2}

\PY{n}{f}\PY{p}{(}\PY{l+m+mi}{5}\PY{p}{)}

[Output]
Decorator arguments: (1, 2, 3)
\end{Verbatim}
\end{frame}


\begin{frame}
  \frametitle{Repeat}
  \begin{Verbatim}[label=\makebox{\url{https://github.com/lucabaldini/cmepda/tree/master/slides/latex/snippets/decorator\_repeat.py}},commandchars=\\\{\}]
\PY{k+kn}{from} \PY{n+nn}{functools} \PY{k+kn}{import} \PY{n}{wraps}

\PY{k}{def} \PY{n+nf}{repeat}\PY{p}{(}\PY{n}{num\PYZus{}times}\PY{p}{)}\PY{p}{:}
    \PY{l+s+sd}{\PYZdq{}\PYZdq{}\PYZdq{} Repeat a function call a given number of times\PYZdq{}\PYZdq{}\PYZdq{}}
    \PY{k}{def} \PY{n+nf}{decorator}\PY{p}{(}\PY{n}{function}\PY{p}{)}\PY{p}{:}
        \PY{n+nd}{@wraps}\PY{p}{(}\PY{n}{function}\PY{p}{)}
        \PY{k}{def} \PY{n+nf}{wrapper}\PY{p}{(}\PY{o}{*}\PY{n}{args}\PY{p}{,} \PY{o}{*}\PY{o}{*}\PY{n}{kwargs}\PY{p}{)}\PY{p}{:}
            \PY{k}{for} \PY{n}{\PYZus{}} \PY{o+ow}{in} \PY{n+nb}{range}\PY{p}{(}\PY{n}{num\PYZus{}times}\PY{p}{)}\PY{p}{:}
                \PY{n}{function}\PY{p}{(}\PY{o}{*}\PY{n}{args}\PY{p}{,} \PY{o}{*}\PY{o}{*}\PY{n}{kwargs}\PY{p}{)}
        \PY{k}{return} \PY{n}{wrapper}
    \PY{k}{return} \PY{n}{decorator}

\PY{c+c1}{\PYZsh{} usage}
\PY{n+nd}{@repeat}\PY{p}{(}\PY{n}{num\PYZus{}times}\PY{o}{=}\PY{l+m+mi}{4}\PY{p}{)}
\PY{k}{def} \PY{n+nf}{greet}\PY{p}{(}\PY{p}{)}\PY{p}{:}
    \PY{k}{print}\PY{p}{(}\PY{l+s+s1}{\PYZsq{}}\PY{l+s+s1}{Hello!}\PY{l+s+s1}{\PYZsq{}}\PY{p}{)}

\PY{n}{greet}\PY{p}{(}\PY{p}{)}

[Output]
Hello!
Hello!
Hello!
Hello!
\end{Verbatim}
\end{frame}


\begin{frame}
  \frametitle{Chaining decorators}
  \begin{Verbatim}[label=\makebox{\url{https://github.com/lucabaldini/cmepda/tree/master/slides/latex/snippets/decorator\_chain.py}},commandchars=\\\{\}]
\PY{k+kn}{from} \PY{n+nn}{decorator\PYZus{}samples} \PY{k+kn}{import} \PY{n}{clocked}\PY{p}{,} \PY{n}{repeat}\PY{p}{,} \PY{n}{print\PYZus{}function\PYZus{}info}

\PY{l+s+sd}{\PYZdq{}\PYZdq{}\PYZdq{} To chain the effect of more than one decorator just stack them above the }
\PY{l+s+sd}{function definition \PYZdq{}\PYZdq{}\PYZdq{}}

\PY{n+nd}{@print\PYZus{}function\PYZus{}info}
\PY{n+nd}{@repeat}\PY{p}{(}\PY{n}{num\PYZus{}times}\PY{o}{=}\PY{l+m+mi}{3}\PY{p}{)}
\PY{n+nd}{@clocked}
\PY{k}{def} \PY{n+nf}{greet}\PY{p}{(}\PY{n}{name}\PY{p}{)}\PY{p}{:}
    \PY{k}{print}\PY{p}{(}\PY{n}{f}\PY{l+s+s1}{\PYZsq{}}\PY{l+s+s1}{Hello \PYZob{}name\PYZcb{}}\PY{l+s+s1}{\PYZsq{}}\PY{p}{)}
    
\PY{n}{greet}\PY{p}{(}\PY{l+s+s1}{\PYZsq{}}\PY{l+s+s1}{Bob}\PY{l+s+s1}{\PYZsq{}}\PY{p}{)}

[Output]
Calling function 'greet'
Positional arguments = ('Bob',)
Keyword arguments = {}
Hello Bob
Function greet executed in 4.000000000000531e-06 s
Hello Bob
Function greet executed in 2.000000000002e-06 s
Hello Bob
Function greet executed in 2.000000000002e-06 s
\end{Verbatim}
\end{frame} 
  
\end{frame}


\end{document}
