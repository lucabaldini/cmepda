\documentclass[9pt]{beamer}
\usetheme{cmepda}

\usepackage[utf8]{inputenc}
\usepackage[T1]{fontenc}


\title{A \LaTeX\ template}
\subtitle{Computing Methods for Experimental Physics and Data Analysis}
\date{Compiled on \today}
\author{Luca Baldini}
\institute[UNIPI and INFN]{Universit\`a and INFN--Pisa}
\email{luca.baldini@pi.infn.it}


\begin{document}


\titleframe


\begin{frame}
  \frametitle{Title}
  \framesubtitle{Subtitle}
  \begin{itemize}
  \item This is a bullet.
    \begin{itemize}
    \item And this is a sub-bullet.
    \end{itemize}
  \item This is another bullet.
  \end{itemize}
  \begin{enumerate}
  \item And you can do numbered lists too\ldots
  \item (Like this.)
  \item<2> All basic beamer overaly mechanisms work.
  \item<2> In normal frame mode \LaTeX\ takes care of the positioning.
  \end{enumerate}
\end{frame}


\begin{frame}
  \frametitle{Code snippets}
  \framesubtitle{An example}

  \begin{Verbatim}[label=\makebox{\url{https://bitbucket.org/lbaldini/programming/src/tip/snippets/hello\_world.py}},commandchars=\\\{\}]
\PY{k}{print}\PY{p}{(}\PY{l+s+s1}{\PYZsq{}}\PY{l+s+s1}{Hello world!}\PY{l+s+s1}{\PYZsq{}}\PY{p}{)}

[Output]
Hello world!
\end{Verbatim}
\end{frame}


\end{document}
