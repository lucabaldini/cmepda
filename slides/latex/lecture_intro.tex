\documentclass[9pt]{beamer}
\usetheme{cmepda}

\usepackage[utf8]{inputenc}
\usepackage[T1]{fontenc}


\title{\scriptsize Computing Methods for Experimental Physics and Data Analysis}
\subtitle{Introduction}
\date{\today}
\author{L. Baldini, G. Lamanna, A. Manfreda, A. Retico, A. Rizzi}
\institute[UNIPI and INFN]{Universit\`a and INFN--Pisa}
\email{luca.baldini@pi.infn.it}


\begin{document}


\titleframe

\begin{frame}
  \frametitle{Goals and prerequisites}
  \begin{itemize}
  \item What is this all about?
    \begin{itemize}
    \item Automating repetitive tasks
    \item Python basics, standard library and scientific ecosystem
    \item Collaborative code development and best practices
    \item Algorithms and data structures
    \item Machine learning
    \item Specific tools for high-energy physics or medical physics
    \end{itemize}
  \item \alert{This is not so much about Python or C++---it is about how to
    write code for effective data analysis}
  \item Will I be a professional data scientist at the end of the semester?
    \begin{itemize}
    \item No, but hopefully you'll be able to poke around and find the right
      tool for the job at hand
    \end{itemize}
  \item What do I need to know in advance?
    \begin{itemize}
    \item Have a vague idea of how a computer operates
    \item If you have ever programmed before that would be
      great!
    \end{itemize}
  \item \alert{This is our first round: we will adjust along the way}
  \end{itemize}
\end{frame}


\begin{frame}
  \frametitle{Basic structure of the course}
  \begin{cmepdapicture}
    \pgfmathsetmacro{\y}{0.90}
    \pgfmathsetmacro{\dy}{0.25}
    \pgfmathsetmacro{\mw}{90pt}
    \node[cmepdadiagblock, name=basic, minimum width=\mw] at (0.5, \y) {
      Basic module\\[2pt](Sep. 19--Oct. 17)
    };
    \node[cmepdadiagblock, name=advanced, minimum width=\mw] at (0.5, \y-\dy) {
      Advanced module\\[2pt](Oct. 20--Nov. 17)
    };
    \node[cmepdadiagblock, name=hep, minimum width=\mw] at (0.25, \y-2*\dy) {
      Fundamental Interactions\\[2pt](Nov. 21--Dec. 19)
    };
    \node[cmepdadiagblock, name=med, minimum width=\mw] at (0.75, \y-2*\dy) {
      Medical Physics\\[2pt](Nov. 21--Dec. 19)
    };
    \draw[arrow] (basic.south) to (advanced.north);
    \draw[arrow, out=-90, in=90] (advanced.south) to (hep.north);
    \draw[arrow, out=-90, in=90] (advanced.south) to (med.north);

    \cmepdaminipagenode[anchor=north]{0.5, 0.20}{\textwidth}{
      \begin{itemize}
      \item Modularity and standard paths:
        \begin{itemize}
        \item Each module is worth 3 credits
        \item 6 credits: basic + advanced
        \item 9 credits: basic + advanced + fundamental interactions
        \item 9 credits: basic + advanced + medical physics
        \end{itemize}
      \end{itemize}
    }
  \end{cmepdapicture}
\end{frame}


\begin{frame}
  \frametitle{Basic module}
  \framesubtitle{L. Baldini and A. Manfreda}
  \begin{itemize}
  \item Collaborative tools
    \begin{itemize}
    \item Version control, development workflow, development platforms
    \end{itemize}
  \item Python basics
    \begin{itemize}
    \item Coding conventions, structuring a package
    \item Variables, native types, functions
    \item The Python standard library
    \end{itemize}
  \item Algorithms and data structures
    \begin{itemize}
    \item Complexity and asymptotic running time
    \item Python data structures and native algorithms
    \end{itemize}
  \item Object-Oriented Programming (OOP)
    \begin{itemize}
    \item Classes, inheritance, composition
    \item Operator overload and emulation of Python builtin types
    \end{itemize}
  \item The Python computing ecosystem
    \begin{itemize}
    \item numpy: arrays, functions, broadasting
    \item Vectorization
    \item Scipy: plotting and fitting
    \item Pandas
    \end{itemize}
  \end{itemize}
\end{frame}


\begin{frame}
  \frametitle{Advanced module}
  \framesubtitle{L. Baldini, G. Lamanna, A. Manfreda, A. Rizzi}
  \begin{itemize}
  \item Advanced code development
    \begin{itemize}
    \item Unit testing, continuous integration, static analysis, documentation
    \end{itemize}
  \item Advanced Python
    \begin{itemize}
    \item Errors, exceptions, iterators and generators, decorators
    \item Profiling and optimization
    \end{itemize}
  \item Parallel computing
    \begin{itemize}
    \item Computer architectures, memory, scaling laws, CPUs and GPUs
    \item Parallel programming: concurrency and parallelism, threading in Python
    \end{itemize}
  \item Machine learning
    \begin{itemize}
    \item Classification and regression: boosted decision trees and
      multilayer perceptrons
    \item Deep learning: neural networks, the keras library
    \item Supervised and unsupervised training, reinforcement learning
    \item Tensorflow
    \end{itemize}
  \end{itemize}
\end{frame}


\begin{frame}
  \frametitle{Fundamental Interactions}
  \framesubtitle{G. Lamanna, A. Rizzi}
  \begin{itemize}
  \item Introduction to C++
    \begin{itemize}
    \item Coding style and organization, declaration of interfaces
    \item Classes: constructors, virtual functions, private and public,
      abstract classes, inheritance
    \item References, pointers, dynamic memory allocation, memory ownership,
      smart pointers
    \item Templates, standard template library
    \item C++11 and C++14: lambda functions, auto variables
    \end{itemize}
  \item More parallel computing
    \begin{itemize}
    \item Cuda and OpenCL
    \item Examples of algorithms for HEP
    \item GPU in HEP Data Analysis
    \end{itemize}
  \item The ROOT data analysis framework
    \begin{itemize}
    \item ROOT toolkit
    \item PyROOT, root-numpy, RDataFrame
    \end{itemize}
  \end{itemize}
\end{frame}


\begin{frame}
  \frametitle{Medical Physics}
  \framesubtitle{A. Retico}
  \begin{itemize}
  \item Medical data processing and feature extraction (python/MATLAB)%
    \begin{itemize}
    \item Tools for handling standard-format medical data (DICOM)
    \item Data anonymization and visualization
    \item Deriving features form images, image segmentation
    \item Data quality control pipelines: outlier removal, dimensionality
      reduction
    \end{itemize}
  \item Data analysis and classification (python/MATLAB)
    \begin{itemize}
    \item Performance evaluations: figures of merit, cross-validation schemes,
      permutation test
    \item Machine-learning and deep-learning tools for segmentation and
      classification
    \item Data augmentation, transfer learning, retrieving localization
      information.
    \end{itemize}
  \end{itemize}
\end{frame}


\begin{frame}
  \frametitle{Logistics}
  \framesubtitle{Timetable and final exam}
  \begin{itemize}
  \item Timetable: 6 hours a week
    \begin{itemize}
    \item Monday, 08:30--11:30 (room O)
    \item Thursday, 08:30--11:30 (room O)
    \end{itemize}
  \item \alert{Lectures are in person!}
  \begin{itemize}
    \item \alert{Streaming over Teams can be accomodated for remote participation
      upon request}
  \end{itemize}
  \item Lectures: 4 hours
    \begin{itemize}
    \item Tentative idea: 2~hour with slides and $2$~hour at the computer
    \item (Actual mileage might vary)
    \end{itemize}
  \item ``Lab'': 2 hours on Thursday morning
    \begin{itemize}
    \item Attack a small project each week in a 2--3~hour session
    \item (Bring your own laptop!)
    \end{itemize}
  \item And, of course, the final exam
    \begin{itemize}
    \item Development of a specific, reasonable-size software project
      (related to the topics covered in the course)
    \item Two-page description of the project and source code made
      available in advance
    \item Oral exam (starting from a presentation of the project)
    \item \alert{New: a few questions on the course material from a pre-compiled list}
    \end{itemize}
  \end{itemize}
\end{frame}

\end{document}
